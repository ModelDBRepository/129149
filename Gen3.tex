\subsection{Point current input}
It is through the mathematical description of point sources of
current that the generalised compartmental model is superior to
the traditional compartmental model. Consider first the
description of exogenous input. The essential difference between
exogenous current input and synaptic current input is that the
contribution of the former is independent of the membrane
potential whereas the contribution of the latter is governed by
the membrane potential at the synapse. In overview, the mechanism
used to describe both is identical.

Let $x_1<x_2<\cdots x_{n-1}$ be the ordered locations of current
input $I_1,\ I_2,\ \cdots\ I_{n-1}$ on the dendritic membrane
lying between the designated nodes $x_0=x_\mathrm{L}$ and
$x_n=x_\mathrm{C}$. The axial current $I_\mathrm{LC}$ flowing from
$x_\mathrm{L}$ to $x_\mathrm{C}$ must now be modified by the
presence of each current source. In the absence of point sources
of current to the dendritic membrane occupying
$[x_\mathrm{L},x_\mathrm{C}]$, $I\mathrm{LC}$ is the axial current
flowing from $x_\mathrm{L}$ towards $x_\mathrm{C}$. On the other
hand, if one allows point sources of current on the dendritic
membrane occupying $[x_\mathrm{L},x_\mathrm{C}]$, Figure
\ref{realcurrent} indicates that current $J_0$ and \textbf{not}
$I_\mathrm{LC}$ flows from $x_\mathrm{L}$ towards $x_\mathrm{C}$
and therefore $I_\mathrm{LC}$ must be corrected by the inclusion
of point current $(J_0-I_\mathrm{LC})$ at $x_\mathrm{L}$. A
similar argument applies to current reaching $x_\mathrm{C}$ from
$x_\mathrm{L}$. The current flowing into $x_\mathrm{C}$ is
$J_{n-1}$ and \textbf{not} $I_\mathrm{LC}$ and therefore the
correction $(I_\mathrm{LC}-J_{n-1})$ must be included as a point
current at $x_\mathrm{C}$.

\begin{figure}[!h]
\centerline{\qquad\begin{mfpic}[0.9][1]{-50}{400}{-30}{50}
\pen{1pt}
\headlen7pt
%
% Sealed cable
\arrow\lines{(0,20),(25,20)}
\arrow\lines{(40,20),(65,20)}
\arrow\lines{(120,20),(145,20)}
\arrow\lines{(160,20),(185,20)}
\arrow\lines{(240,20),(265,20)}
\arrow\lines{(280,20),(305,20)}
%
%
\dotspace=4pt
\dotsize=2pt
\dotted\lines{(75,20),(110,20)}
\dotted\lines{(195,20),(230,20)}
%
% Nodes on sealed cable
\tlabel[cc](0,20){\large $\bullet$}
\tlabel[cc](40,20){\large $\bullet$}
\tlabel[cc](120,20){\large $\bullet$}
\tlabel[cc](160,20){\large $\bullet$}
\tlabel[cc](240,20){\large $\bullet$}
\tlabel[cc](280,20){\large $\bullet$}
\tlabel[cc](320,20){\large $\bullet$}
%
% Points on sealed cable
\tlabel[br](0,30){$x_\mathrm{L}=x_0$}
\tlabel[bc](40,30){$x_1$}
\tlabel[bc](120,30){$x_{k-1}$}
\tlabel[bc](160,30){$x_k$}
\tlabel[bc](240,30){$x_{n-2}$}
\tlabel[bc](280,30){$x_{n-1}$}
\tlabel[bl](320,30){$x_n=x_\mathrm{C}$}
%
\tlabel[cc](20,10){$J_0$}
\tlabel[cc](60,10){$J_1$}
\tlabel[cc](140,10){$J_{k-1}$}
\tlabel[cc](180,10){$J_k$}
\tlabel[cc](260,10){$J_{n-2}$}
\tlabel[cc](300,10){$J_{n-1}$}
%
% Sealed cable
\arrow\lines{(40,10),(40,-10)}
\tlabel[tc](40,-15){\textsf{$I_1$}}
\arrow\lines{(120,10),(120,-10)}
\tlabel[tc](120,-15){\textsf{$I_{k-1}$}}
\arrow\lines{(160,10),(160,-10)}
\tlabel[tc](160,-15){\textsf{$I_k$}}
\arrow\lines{(240,10),(240,-10)}
\tlabel[tc](240,-15){\textsf{$I_{n-2}$}}
\arrow\lines{(280,10),(280,-10)}
\tlabel[tc](280,-15){\textsf{$I_{n-1}$}}
\end{mfpic}}
\centering
\parbox{4in}{\caption{\label{realcurrent} Configuration
of point current input to the length of dendritic membrane between
$x_\mathrm{L}$ and $x_\mathrm{C}$.}}
\end{figure}

If current $J_k$ flows from $x_k$ to $x_{k+1}$, then the aim of
this analysis is to determine the corrections
$(J_0-I_\mathrm{LC})$ and $(I_\mathrm{LC}-J_{n-1})$ that must be
applied to $I_\mathrm{LC}$ at $x_\mathrm{L}$ and $x_\mathrm{C}$
respectively as a consequence of the point current input to the
dendritic membrane occupying $[x_\mathrm{L},x_\mathrm{C}]$.
Kirchhoff's laws applied to the section of dendrite illustrated in
Figure \ref{realcurrent} give
\begin{equation}\label{pc2}
J_0=J_1+I_1\,,\quad\cdots\quad J_{k-1}=J_k+I_k\,,\quad\cdots\quad
J_{n-2}=J_{n-1}+I_{n-1}
\end{equation}
where
\begin{equation}\label{pc3}
J_k = \frac{\pi g_\mathrm{A}r_k r_{k+1}(V_k-V_{k+1})}{x_{k+1}-x_k}
\,,\qquad k=0,1,\cdots,(n-1)\,.
\end{equation}
Bearing in mind that $V_0=V_\mathrm{L}$ and that
$V_n=V_\mathrm{C}$, it follows directly from $J_{k-1}=J_k+I_k$
that the potentials $V_1,\cdots, V_{n-1}$ satisfy the equations
\begin{equation}\label{pc4}
\frac{\pi g_\mathrm{A}r_{k-1} r_k(V_{k-1}-V_k)}{x_k-x_{k-1}}
-\frac{\pi g_\mathrm{A}r_k r_{k+1}(V_k-V_{k+1})}
{x_{k+1}-x_k}=I_k\,,\qquad k=1,\cdots,n-1\,,
\end{equation}
which on division by $\pi g_\mathrm{A} r_k$ yields
\begin{equation}\label{pc4}
\frac{r_{k-1} V_{k-1}}{x_k-x_{k-1}}
-\Big(\frac{r_{k-1} V_k}{x_k-x_{k-1}}+
\frac{r_{k+1}V_k}{x_{k+1}-x_k}\Big)
+\frac{r_{k+1}V_{k+1}}{x_{k+1}-x_k}
=\frac{I_k}{\pi g_\mathrm{A} r_k}\,,\qquad k=1,\cdots,n-1\,.
\end{equation}
It is a matter of straight forward algebra to verify that
\[
\frac{r_{k-1} V_k}{x_k-x_{k-1}}+\frac{r_{k+1}V_k}{x_{k+1}-x_k}
=\frac{r_k V_k (x_{k+1}-x_{k-1})}{(x_k-x_{k-1})(x_{k+1}-x_k)}\,,
\]
and when the right hand side of this equation is used to simplify
the middle term in equation (\ref{pc4}), the result may be
expressed in the form
\begin{equation}\label{pc5}
\frac{\big(\,r_{k-1} V_{k-1}-r_k V_k\,\big)}{x_k-x_{k-1}}
-\frac{\big(\,r_k V_k- r_{k+1}V_{k+1}\,\big)}{x_{k+1}-x_k}
=\frac{I_k}{\pi g_\mathrm{A} r_k}\,,\qquad k=1,\cdots,n-1\,.
\end{equation}
Most importantly, the validity of equation (\ref{pc5}) is
independent of the specific nature of $I_k$, that is, it is valid
for both exogenous current for which $I_k$ is independent of $V_k$
and synaptic current for which $I_k=g_k(t)(V_k-E_k)$ where $E_k$
is the reversal potential of the ionic species of the synapse. In
all cases, the task is to compute the corrections
$(J_0-I_\mathrm{LC})$ and $(I_\mathrm{LC}-J_{n-1})$, and to
express these corrections in terms of the potentials
$V_\mathrm{L}$ and $V_\mathrm{C}$.

\subsection{Exogenous input}
The application of identity (\ref{pc5}) to exogenous current is
examined first. The aim of this section is to use identity
(\ref{pc5}) to establish the corrections
\begin{equation}\label{ei1}
J_0-I_\mathrm{LC}=\sum_{k=1}^{n-1}\,\frac{r_\mathrm{L}}{r_k}\,
\frac{x_\mathrm{C}-x_k}{x_\mathrm{C}-x_\mathrm{L}}\;I_k\,,\qquad
I_\mathrm{LC}-J_{n-1}=\sum_{k=1}^{n-1}\,\frac{r_\mathrm{C}}{r_k}\,
\frac{x_k-x_\mathrm{L}}{x_\mathrm{C}-x_\mathrm{L}}\;I_k
\end{equation}
to be applied at $x_\mathrm{L}$ and $x_\mathrm{C}$ respectively.
These expressions make explicit how input $I_k$ is partitioned
between nodes $x_\mathrm{L}$ and $x_\mathrm{C}$ taking account of
its location within the segment and the geometry of the segment.
To facilitate the derivation of these results, it is convenient to
define
\begin{equation}\label{ei2}
\psi_k=\frac{r_k V_k- r_{k+1}V_{k+1}}{x_{k+1}-x_k}\,.
\end{equation}
By replacing $I_k/r_k$ from identity (\ref{pc5}), it follows that
\[
\begin{array}{rcl}
\ds\sum_{k=1}^{n-1}\,\frac{r_0}{r_k}\,
\frac{x_n-x_k}{x_n-x_0}\;I_k & = & \ds\frac{\pi g_\mathrm{A} r_0}
{x_n-x_0}\,\sum_{k=1}^{n-1}\,\big(\,x_n-x_k\,\big)
\big(\,\psi_{k-1}-\psi_k\,\big)\\[10pt]
& = & \ds\frac{\pi g_\mathrm{A} r_0}{x_n-x_0}\,\Big[\,
\sum_{k=0}^{n-2}\,\big(\,x_n-x_{k+1}\,\big)\,\psi_k
-\sum_{k=1}^{n-1}\,\big(\,x_n-x_k\,\big)\psi_k\,\Big]\\[10pt]
& = & \ds\frac{\pi g_\mathrm{A} r_0}{x_n-x_0}\,\Big[\,
\big(\,x_n-x_1\,\big)\,\psi_0-\sum_{k=1}^{n-2}\,
\big(\,x_{k+1}-x_k\,\big)\psi_k-\big(\,x_n-x_{n-1}\,\big)\,
\psi_{n-1}\,\Big]\\[10pt]
& = & \ds\frac{\pi g_\mathrm{A} r_0}{x_n-x_0}\,\Big[\,
\big(\,x_n-x_1\,\big)\,\psi_0-\sum_{k=1}^{n-1}\,
\big(\,r_k V_k-r_{k+1}V_{k+1}\,\big)\,\Big]\\[10pt]
& = & \ds\frac{\pi g_\mathrm{A} r_0}{x_n-x_0}\,\Big[\,
\frac{(x_n-x_1)}{(x_1-x_0)}\,\big(\,r_0 V_0-r_1 V_1\,\big)
-\big(r_1 V_1-r_n V_n\big)\,\Big]\\[10pt]
& = & \ds\frac{\pi g_\mathrm{A} r_0}{x_n-x_0}\,\Big[\,
\frac{(x_n-x_0)}{(x_1-x_0)}\,\big(\,r_0 V_0-r_1 V_1\,\big)
+r_n V_n-r_0 V_0\,\Big]\\[10pt]
& = & \ds\frac{\pi g_\mathrm{A} r_0 r_n (V_n-V_0)}{x_n-x_0}
+\frac{\pi g_\mathrm{A} r_0 r_1\big(\,V_0-V_1\,\big)}
{x_1-x_0}\\[10pt]
& = & J_0-I_\mathrm{LC}\,.
\end{array}
\]
The second identity is obtained directly from the first by a piece
of straight forward algebra based on the observation that
$J_0=J_{n-1}+I_1+I_2+\cdots+I_{n-1}$.

For example, in the special case of a single point input $I_1$ at
point $x_1$ between the designated nodes $x_\mathrm{L}$ and
$x_\mathrm{C}$, if follows directly from (\ref{ei1}) that
\begin{equation}\label{ei9}
J_0-I_\mathrm{LC}=\frac{r_\mathrm{L}}{r_1}\,
\frac{x_\mathrm{C}-x_1}{x_\mathrm{C}-x_\mathrm{L}}\;I_1\,,\qquad
I_\mathrm{LC}-J_1=\frac{r_\mathrm{C}}{r_1}\,
\frac{x_1-x_\mathrm{L}}{x_\mathrm{C}-x_\mathrm{L}}\;I_1\,.
\end{equation}
If, in addition, the segment is uniform then $I_1$ is partitioned
between the left hand and right hand endpoints of the segment in
the proportion in which the position of the input divides the
distance between the designated nodes.

\subsection{Synaptic input}
To appreciate how synaptic input differs from exogenous input,
consider first the case of a single synapse of strength $g(t)$ at
$x_\mathrm{S}\in\mathcal{L}$. In this simple case, the currents
$J_0$ and $J_1$ are respectively
\begin{equation}\label{si1}
J_0 = \frac{\pi g_\mathrm{A}r_\mathrm{L} r_\mathrm{S}
(V_\mathrm{L}-V_\mathrm{S})}{x_\mathrm{S}-x_\mathrm{L}}\,,
\qquad
J_1 = \frac{\pi g_\mathrm{A}r_\mathrm{S} r_\mathrm{C}
(V_\mathrm{S}-V_\mathrm{C})}{x_\mathrm{C}-x_\mathrm{S}}\,,
\qquad J_0=J_1+g(t)(V_\mathrm{S}-E_\mathrm{S})
\end{equation}
where $V_\mathrm{S}$ is the membrane potential at the synapse,
$E_\mathrm{S}$ is the reversal potential for the ionic species of
the synapse and $r_\mathrm{S}$ is the radius of the dendrite at
the position of the synapse. It is easy to show from equations
(\ref{si1}) that the required modification to the original core
current at $x_\mathrm{L}$ and at $x_\mathrm{C}$ are respectively
\begin{equation}\label{si2}
\hskip-8pt\begin{array}{rcl}
J_0-I_\mathrm{LC}& = & \ds \frac{\pi g(t) g_\mathrm{A}r_\mathrm{L}}
{x_\mathrm{C}-x_\mathrm{L}}\,\Big[
\frac{r_\mathrm{L}(x_\mathrm{C}-x_\mathrm{S})^2
(V_\mathrm{L}-E_\mathrm{S})+r_\mathrm{C}(x_\mathrm{S}-x_\mathrm{L})
(x_\mathrm{C}-x_\mathrm{S})(V_\mathrm{C}-E_\mathrm{S})}
{g(t)(x_\mathrm{C}-x_\mathrm{S})(x_\mathrm{S}-x_\mathrm{L})+
\pi g_\mathrm{A} r^2_\mathrm{S}(x_\mathrm{C}-x_\mathrm{L})}\,\Big]\,,\\[10pt]
I_\mathrm{LC}-J_1 & = & \ds\frac{\pi g(t) g_\mathrm{A}r_\mathrm{C}}
{x_\mathrm{C}-x_\mathrm{L}}\,\Big[
\frac{r_\mathrm{L}(x_\mathrm{S}-x_\mathrm{L})
(x_\mathrm{C}-x_\mathrm{S})(V_\mathrm{L}-E_\mathrm{S})
+r_\mathrm{C}(x_\mathrm{S}-x_\mathrm{L})^2
(V_\mathrm{C}-E_\mathrm{S})}{g(t)(x_\mathrm{C}-x_\mathrm{S})
(x_\mathrm{S}-x_\mathrm{L})+\pi g_\mathrm{A} r^2_\mathrm{S}
(x_\mathrm{C}-x_\mathrm{L})}\,\Big]\,.
\end{array}
\end{equation}
In particular, it is clear that the modification to the core
current $I_\mathrm{LC}$ can be characterised exactly by the
addition of currents at $x_\mathrm{L}$ and $x_\mathrm{C}$.
Although this methodology can be continued for many different
synapses in the interval $(x_\mathrm{L},x_\mathrm{C})$, it is
clear that this approach, when used to describe the effect of many
synapses, will lead to an unacceptable level of complexity in the
representation of their effect. What is required are approximate
but yet tractable expressions for the modifications
$J_0-I_\mathrm{LC}$ and $I_\mathrm{LC}-J_{n-1}$ in the axial
current at $x_\mathrm{L}$ and $x_\mathrm{C}$. Moreover, these
expressions should recognise that synaptic activity changes the
local potential distribution.

Towards this end, let the synapse at node $x_k\in\mathcal{L}$ have
conductance $g_k(t)$ and reversal potential $E_k$, then the
current supplied by that synapse is $I_k=g_k(t)(V_k-E_k)$ where
$V_k$ is the membrane potential at location $x_k$ and time $t$. To
avoid the complexity alluded to, but yet take advantage of
formulae (\ref{ei1}), the membrane potential
\[
V(x,t)=\frac{V_\mathrm{L}\,r_\mathrm{L}\,(x_\mathrm{C}-x)+
V_\mathrm{C}\,r_\mathrm{C}(x-x_\mathrm{L})}{r_k\,
(x_\mathrm{C}-x_\mathrm{L})}
\]
is used in the first instance to estimate the synaptic current
using the formula
\begin{equation}\label{si3}
\begin{array}{rcl}
I_k=g_k(t)\big(V_k-E_k\big) & = & \ds g_k(t)\Big(
\frac{V_\mathrm{L}\,r_\mathrm{L}\,(x_\mathrm{C}-x_k)+
V_\mathrm{C}\,r_\mathrm{C}(x_k-x_\mathrm{L})}{r_k\,
(x_\mathrm{C}-x_\mathrm{L})}-E_k\Big)\\[10pt]
& = & \ds g_k(t)\,\frac{r_\mathrm{L}}{r_k}\,\frac{x_\mathrm{C}-x_k}
{x_\mathrm{C}-x_\mathrm{L}}\,\big(V_\mathrm{L}-E_k\big)
+g_k(t)\,\frac{r_\mathrm{C}}{r_k}\,\frac{x_k-x_\mathrm{L}}
{x_\mathrm{C}-x_\mathrm{L}}\,\big(V_\mathrm{C}-E_k\big)\,.
\end{array}
\end{equation}
This expression for $I_k$ can now be used in formulae (\ref{ei1})
to conclude that the effect of synaptic input at $x_1,\cdots,
x_{n-1}$ may be described by the addition of current
\begin{equation}\label{si4}
\begin{array}{rcl}
J_0-I_\mathrm{LC} & = & \ds V_\mathrm{L}\,\sum_{k=1}^{n-1}\,g_k(t)\,
\Big(\,\frac{r_\mathrm{L}}{r_k}\,\frac{x_\mathrm{C}-x_k}
{x_\mathrm{C}-x_\mathrm{L}}\,\Big)^2\\[10pt]
&&\qquad\ds+\; V_\mathrm{C}\,\sum_{k=1}^{n-1}\,g_k(t)\,
\frac{r_\mathrm{L} r_\mathrm{C}}{r^2_k}\,
\frac{(x_\mathrm{C}-x_k)(x_k-x_\mathrm{L})}
{(x_\mathrm{C}-x_\mathrm{L})^2}-\sum_{k=1}^{n-1}\,
\frac{r_\mathrm{L}}{r_k}\,
\frac{x_\mathrm{C}-x_k}{x_\mathrm{C}-x_\mathrm{L}}\;g_k(t) E_k\,,
\end{array}
\end{equation}
at designated point $x_\mathrm{L}$ and by the addition of current
\begin{equation}\label{si5}
\begin{array}{rcl}
I_\mathrm{LC}-J_{n-1} & = & \ds V_\mathrm{L}\,
\sum_{k=1}^{n-1}\,g_k(t)\,\frac{r_\mathrm{L}r_\mathrm{C}}{r^2_k}\,
\frac{(x_k-x_\mathrm{L})(x_\mathrm{C}-x_k)}
{(x_\mathrm{C}-x_\mathrm{L})^2}\\[10pt]
&&\qquad\ds+\;V_\mathrm{C}\,\sum_{k=1}^{n-1}\,g_k(t)\,
\Big(\frac{r_\mathrm{C}}{r_k}\,\frac{x_k-x_\mathrm{L}}
{x_\mathrm{C}-x_\mathrm{L}}\Big)^2-\sum_{k=1}^{n-1}\,
\frac{r_\mathrm{C}}{r_k}\,
\frac{x_k-x_\mathrm{L}}{x_\mathrm{C}-x_\mathrm{L}}\;g_k(t) E_k
\end{array}
\end{equation}
at designated point $x_\mathrm{C}$.

\subsection{The model differential equations}
Suppose that the neuron is partitioned into $m$ compartments where
the membrane potential at the designated node of the $k^{th}$
compartment is $V_k(t)$ and let
\begin{equation}\label{gmde1}
V(t)=\big[\,V_1(t),V_1(t),\cdots,V_m(t)\,]^\mathrm{T}\,.
\end{equation}
The fundamental difference between the generalised and traditional
compartmental models lies in the specification of the
transmembrane current. In the traditional model, the specification
of the transmembrane current falling on a compartment depends only
on the membrane potential at the designated node of the
compartment itself. By contrast, in the generalised model the
mathematical specification of the transmembrane current falling on
a compartment depends not only on the membrane potential at the
designated node of the compartment, but also on the membrane
potential at the designated nodes of the neighbouring
compartments. For example, the capacitative current in the
generalised model is expressed as a linear combination of the
derivative of the membrane potential at the designated node of the
compartment and the derivative of the membrane potential at the
nodes of the neighbouring compartments. Similarly, the intrinsic
voltage-dependent currents and synaptic currents are expressed as
linear combinations of functions of the membrane potential at the
designated node of the compartment and those of the neighbouring
compartments.

It follows immediately from these observations that $V(t)$, the
column vector of membrane potentials, satisfies the ordinary
differential equations
\begin{equation}\label{gmde2}
F^\mathrm{C}\,\frac{dV}{dt}+G^\mathrm{IVDC}(V)\,V+G^\mathrm{SYN}(t)\,V
+I(t)=AV
\end{equation}
where the constant matrix $F^\mathrm{C}$ replaces the diagonal
matrix $D^\mathrm{C}$ in expression (\ref{mde3}), the matrix
$G^\mathrm{IVDC}(V)$ of intrinsic voltage-dependent conductances
replaces the diagonal matrix $D^\mathrm{IVDC}(V)$ in (\ref{mde3}),
the matrix $G^\mathrm{SYN}(t)$ of synaptic conductances replaces
the diagonal matrix $D^\mathrm{SYN}(t)$ in (\ref{mde3}) and $I(t)$
is a column vector of exogenous currents. On the other hand, the
conductance matrix $A$ is identical to that of the traditional
compartmental model.

Equation (\ref{gmde2}) is now integrated over the interval
$[t,t+h]$ to get
\begin{equation}\label{gmde3}
F^\mathrm{C}\,\big[\,V(t+h)-V(t)\,\big]+
\int_t^{t+h}\,\big[\,G^\mathrm{IVDC}(V)+
G^\mathrm{SYN}(t)\,\big]\,V(t)\,dt
+\int_t^{t+h}\,I(t)=A\int_t^{t+h}\,V(t)\,dt\,.
\end{equation}
The trapezoidal quadrature is used to replace each integral in
equation (\ref{gmde3}) with the exception of the integral of
intrinsic voltage-dependent current, which is replaced by the
midpoint quadrature. The result of this calculation is
\begin{equation}\label{gmde4}
\begin{array}{l}
\ds F^\mathrm{C}\,\big[\,V(t+h)-V(t)\,\big]+
h\,G^\mathrm{IVDC}(V(t+h/2))\,V(t+h/2)\\[10pt]
\quad\ds+\;\frac{h}{2}\,\Big[\,G^\mathrm{SYN}(t+h)V(t+h)
+G^\mathrm{SYN}(t)V(t)\,\Big]+\frac{h}{2}
\Big[\,I(t+h)+I(t)\,\Big]\\[10pt]
\qquad\ds = \frac{h}{2}\,\Big[\,A V(t+h)+AV(t)\,\Big]+O(h^3)\,.
\end{array}
\end{equation}
On taking account of the fact that
\[
V(t+h/2)=\frac{1}{2} \,\Big[\,V(t+h)+V(t)\,\Big]+O(h^2)\,,
\]
equation (\ref{mde5}) may be reorganised to give
\begin{equation}\label{mde6}
\begin{array}{l}
\ds \Big[\,2 F^\mathrm{C}-hA+h\,G^\mathrm{SYN}(t+h)
+h\,G^\mathrm{IVDC}(V(t+h/2))\,\Big]\,V(t+h) = \\[10pt]
\qquad\ds \Big[\,2 F^\mathrm{C}+hA+h G^\mathrm{SYN}(t)
-h\,G^\mathrm{IVDC}(V(t+h/2))\,\Big]\,V(t)
-h\Big[\,I(t+h)+I(t)\,\Big]
\end{array}
\end{equation}
when the error structure is ignored. The detailed computation of
$G^\mathrm{IVDC}(V(t+h/2))$ is determined entirely by the
structure of the auxiliary equations. In the case of
Hodgkin-Huxley like channels, it is standard knowledge that
$G^\mathrm{IVDC}(V(t+h/2))$ can be computed to adequate accuracy
from $V(t)$ and the differential satisfied by the auxiliary
variables (Lindsay \emph{et al.}, \cite{Lindsay01a}).

The inference to be drawn from the fact that matrix $2
F^\mathrm{C}-hA+h\,G^\mathrm{SYN}(t+h)
+h\,G^\mathrm{IVDC}(V(t+h/2))$ is not more complex than $A$ itself
is that the numerical complexity of the mathematical problem posed
by the generalised compartmental model is identical to that posed
by the traditional compartmental model.
