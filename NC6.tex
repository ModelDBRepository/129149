\addtocontents{toc}{\vskip8pt}
\addtocontents{toc}{\bf\hskip18pt References\hfill \thepage}
\addtocontents{toc}{\vskip4pt}

\begin{thebibliography}{99}

\bibitem[1997]{Bower97} Bower JM and Beeman D (1997) The
book of GENESIS. 2nd ed. NY Telos.

\bibitem[1997]{Hines97} Hines M and Carnevale N (1997) The
NEURON simulation environmment. \emph{Neural Computation}
9:1179-1209.

\bibitem[1952]{Hodgkin52} Hodgkin AL and Huxley AF (1952) A
quantitative description of membrane current and its application
to conduction and excitation in nerve. \emph{Journal of
Physiology} 117:500-544.

\bibitem[2003]{Lindsay03} Lindsay KA, Rosenberg JR and
Tucker G (2003). Analytical and numerical construction of
equivalent cables. \emph{Mathematical Biosciences} 184:137-164.

\bibitem[1999]{Oram99} Oram MW, Wiener MC, Lestienne R, Richmond
BJ (1999) Stochastic nature of precisely timed spike patterns in
visual system neuronal responses. \emph{Journal of
Neurophysiology} 81:3021-3033.

\bibitem[2003]{Poirazi03} Poriazi P, Brannon T, and Mel BW (2003).
Pyramidal neuron as two-layer neural network \emph{Neuron}
37:989-999.

\bibitem[1964]{Rall64} Rall W (1964) Theoretical significance
of dendritic trees and motoneuron input-output relations. In
\emph{Neural Theory and Modelling.} R.F.Reiss  (ed.). Stanford
University Press, Stanford CA.

\bibitem[1998]{Segev98} Segev I and Burke RE (1998)
Compartmental models of complex neurons. In \emph{Methods in
Neuronal Modeling - from ions to networks} 2nd Edition. Koch C and
Segev I (eds.). Ch.3, pp 93-136. MIT Press, MA.

\end{thebibliography}

\pagebreak[4]

\textbf{Appendix 1 -- Numerical estimation of perturbations to axial current}

The example in Subsection \ref{stage1} demonstrates that synaptic
and exogenous input do not act independently. This means that both
types of point process input must be treated simultaneously in the
construction of the equations to determine the perturbations
$I_\mathrm{P}$ and $I_\mathrm{D}$ of the axial current. The
equations for the perturbations in axial current are constructed
by replacing $I_k$ in equations (\ref{syn1b}, \ref{syn4} and
\ref{syn5}) by $I_\mathrm{PD}+\widehat{I}_k$ where $\widehat{I}_k$
is the perturbation to $I_k$. If $\lambda=\lambda_k$ is the site
of an exogenous input then the appropriate equation for the
perturbed currents is
\begin{equation}\label{syn17a}
\widehat{I}_k-\widehat{I}_{k+1}=\mathcal{I}_k(t)\,,
\end{equation}
whereas if $\lambda=\lambda_k$ is the site of a synapse with
conductance $g_k(t)$, the appropriate equation is
\begin{equation}\label{syn17b}
\widehat{I}_k-\widehat{I}_{k+1}+\frac{g_k h}{\pi g_\mathrm{A}}
\sum_{j=1}^k \frac{(\lambda_j-\lambda_{j-1})}{r_{j-1}\,r_j}\,
\widehat{I}_j = \mathcal{I}_k(t)
\end{equation}
where the current $\mathcal{I}_k(t)$ is defined by the formula
\begin{equation}\label{syn17c}
\mathcal{I}_k(t)=g_k(t)\,\Big[\,(1-\lambda_k)\frac{r_\mathrm{P}}{r_k}
\,V_\mathrm{P}+\lambda_k\frac{r_\mathrm{D}}{r_k}\, V_\mathrm{D}
-E_k\,\Big]\,.
\end{equation}
The derivation of equation (\ref{syn17b}) takes advantage of the
identity
\[
\sum_{j=1}^k \,\frac{(\lambda_j-\lambda_{j-1})}{r_{j-1}\,r_j}
=\frac{\lambda_k}{r_\mathrm{P}\,r_k}\,,
\]
which can be established by induction. Note that expression
(\ref{syn17c}) for $\mathcal{I}_k(t)$ when $\lambda=\lambda_k$ is
a synapse is precisely the current that would be expected to flow
at the synapse if the distribution of potential on the segment was
described by expression (\ref{mp3}). Finally, equation
(\ref{syn5}) simplifies to
\begin{equation}\label{syn17d}
\sum_{j=1}^{n+1}
\frac{(\lambda_j-\lambda_{j-1})r_\mathrm{P}\,r_\mathrm{D}}
{r_{j-1}\,r_j}\,\widehat{I}_j = 0
\end{equation}
where the constant multiplier $r_\mathrm{P}\,r_\mathrm{D}$ has
been added without loss to make the coefficients of this equation
comparable to those appearing in the first $n$ equations.
Equations (\ref{syn17a},\ref{syn17b} and \ref{syn17d}) may be
represented compactly in matrix notation by
\begin{equation}\label{syn18}
A\,\widehat{I}+GC\,\widehat{I}=\mathcal{I}
\end{equation}
where $\widehat{I}=[\widehat{I}_1,\ \cdots\ ,\widehat{I}_{n+1}
]^\mathrm{T}$ is the $(n+1)$ dimensional column vector of
perturbations in axial current, $\mathcal{I}=[\mathcal{I}_1,\
\cdots\ , \mathcal{I}_n,0\,]^\mathrm{T}$ and $A$ is the
$(n+1)\times(n+1)$ matrix
\begin{equation}\label{syn19}
\left[\begin{array}{ccccccc}
1 & \hskip-9pt-1 &  0 & \cdots & \cdots & 0 \\[5pt]
0 &  1 & \hskip-9pt-1 & \cdots & \cdots & 0 \\[5pt]
0 &  0 &  1 & \cdots & \cdots & 0 \\[5pt]
\cdots & \cdots & \cdots & \cdots & \cdots & \cdots \\[5pt]
0 & 0 & 0 & \cdots & 1 & \hskip-9pt-1 \\[5pt]
\ds\frac{\lambda_1 r_\mathrm{P}r_\mathrm{D}}{r_0 r_1} &
\ds\frac{(\lambda_2-\lambda_1)r_\mathrm{P}r_\mathrm{D}}{r_1 r_2} &
\ds\frac{(\lambda_3-\lambda_2)r_\mathrm{P}r_\mathrm{D}}{r_2 r_3} &
\cdots &
\ds\frac{(\lambda_n-\lambda_{n-1})r_\mathrm{P}r_\mathrm{D}}{r_{n-1}
r_n} & \ds\frac{(1-\lambda_n)r_\mathrm{P}r_\mathrm{D}} {r_n
r_{n+1}}
\end{array}\right]\,.
\end{equation}
Briefly, $G$ is an $(n+1)\times(n+1)$ diagonal matrix in which the
$(k,k)$ entry is zero if $\lambda_k$ is the site of an exogenous
input and takes the value $g_k(t)$ if $\lambda_k$ is the site of a
synapse. The $(n+1,n+1)$ entry of $G$ is always zero. The matrix
$C$ is a lower triangular matrix of type $(n+1)\times(n+1)$ in
which all the nonzero entries in the $k^{th}$ column take the
value $(\lambda_k-\lambda_{k-1})/(\pi g_\mathrm{A}r_{k-1}\,r_k)$.

\textbf{Multiple point inputs}

To take account of the influence of the matrix $GC$ in the
solution of equation (\ref{syn18}), the algorithm
\begin{equation}\label{syn20}
A\widehat{I}^{(m+1)}=\mathcal{I}-GC\widehat{I}^{(m)}
\end{equation}
is iterated with initial condition
$A\widehat{I}^{(0)}=\mathcal{I}$. Although it can be demonstrated
that the matrix $A$ has a simple closed form expression for its
inverse, it is not (numerically) efficient to use this expression
to solve equation (\ref{syn20}). Instead, we observe that $A$ has
an $LU$ factorisation in which $U$ is the $(n+1)\times(n+1)$ upper
triangular matrix with ones everywhere in the main diagonal,
negative ones everywhere in the super-diagonal and zero everywhere
else, and $L$ is the $(n+1)\times(n+1)$ lower triangular matrix
\begin{equation}\label{syn21}
\left[\begin{array}{ccccccc}
1 & 0 & 0 & 0 & \cdots & \cdots & 0 \\[5pt]
0 & 1 & 0 & 0 & \cdots & \cdots & 0 \\[5pt]
0 & 0 & 1 & 0 & \cdots & \cdots & 0 \\[5pt]
\cdots & \cdots & \cdots & \cdots & \cdots & \cdots & \cdots \\[5pt]
\ds\frac{\lambda_1\,r_\mathrm{P}}{r_1} &
\ds\frac{\lambda_2\,r_\mathrm{P}}{r_2} &
\ds\frac{\lambda_3\,r_\mathrm{P}}{r_3} &
\ds\frac{\lambda_4\,r_\mathrm{P}}{r_4} & \cdots &
\ds\frac{\lambda_n\,r_\mathrm{P}}{r_n} & 1
\end{array}\right]\,.
\end{equation}
Since $\mathcal{I}$ is a linear combination of $V_\mathrm{P}$,
$V_\mathrm{D}$ and a voltage independent term, then the solution
to equation (\ref{syn20}) has general representation
\begin{equation}\label{syn22}
\widehat{I} = \phi_1(t)V_\mathrm{P} +
\phi_2(t)V_\mathrm{D}+\phi_3(t)
\end{equation}
where $\phi_1(t)$, $\phi_2(t)$ and $\phi_3(t)$ satisfy
\begin{equation}\label{syn23}
\begin{array}{rcl}
A\,\phi_1 & = & \Big[\,g_1(1-\lambda_1)\ds\frac{r_\mathrm{P}}{r_1}
,\ \cdots \ ,\
g_n(1-\lambda_n)\ds\frac{r_\mathrm{P}}{r_n},0\,\Big]^\mathrm{T}
-GC\,\phi_1\,,\\[10pt]
A\,\phi_2 & = & \Big[\,g_1\lambda_1\ds\frac{r_\mathrm{D}}{r_1}
,\ \cdots \ ,\ g_n\lambda_n\ds\frac{r_\mathrm{D}}{r_n},0\,\Big]^\mathrm{T}-GC\,\phi_2\,,\\[10pt]
A\,\phi_3 & = & -\Big[\,g_1E_1,\ \cdots \ ,\
g_nE_n,0\,\Big]^\mathrm{T}-GC\,\phi_3\,.
\end{array}
\end{equation}
The equations (\ref{syn23}) for $\phi_1(t)$, $\phi_2(t)$ and
$\phi_3(t)$ may be solved easily by an iterative procedure based
on the sparse $LU$ factorisation of $A$. If the conductances
$g_1,\ \cdots ,\ g_n$ are sufficiently small, the solution of
equations (\ref{syn23}) is well approximated by ignoring the
second term on the right hand side or equations (\ref{syn23}).
This approximation is equivalent to using the partitioning rule
(\ref{potc3}) in combination with formula (\ref{mp3}) for the
membrane potential.

\textbf{Special case of exogenous input}

If $\lambda_1,\ \cdots\ ,\lambda_n$ are sites of exogenous input
$\mathcal{I}_1,\ \cdots \,\mathcal{I}_n$ then $G=0$ in equation
(\ref{syn20}) and $\mathcal{I}$ is the vector of exogenous
currents. In this case, expressions (\ref{ei1}) for $I_\mathrm{P}$
and $I_\mathrm{D}$ are obtained immediately as the first and last
entries in the solution $\widehat{I}$ of equation
$A\,\widehat{I}=LU\,\widehat{I}=\mathcal{I}$.

\textbf{Appendix 2 -- The partitioning of capacitative current on tapered
cylinders}

Recall from expressions (\ref{potc3}) that the contributions made
to the proximal and distal perturbations to the axial current as a
consequence of capacitative transmembrane current on a tapered
segment with membrane of variable specific capacitance are
respectively
\begin{equation}\label{dtc13}
\begin{array}{rcl}
I^\mathrm{\,cap}_\mathrm{P} & = & 2\pi\, r_\mathrm{P} h
\ds\Big[r_\mathrm{P}\frac{dV_\mathrm{P}}{dt}
\int_0^1\frac{(1-\lambda)^2 c_\mathrm{M}(\lambda)\,d\lambda}
{(1-\lambda)\,r_\mathrm{P}+\lambda\,r_\mathrm{D}}
+r_\mathrm{D}\frac{dV_\mathrm{D}}{dt} \int_0^1
\frac{\lambda(1-\lambda)c_\mathrm{M}(\lambda)\,d\lambda}
{(1-\lambda)\,r_\mathrm{P}+\lambda\,r_\mathrm{D}}\Big],\\[12pt]
-I^\mathrm{\,cap}_\mathrm{D} & = & 2\pi\, r_\mathrm{D} h
\ds\Big[r_\mathrm{P}\frac{dV_\mathrm{P}}{dt}\int_0^1\,
\frac{\lambda(1-\lambda)c_\mathrm{M}(\lambda)\,d\lambda}
{(1-\lambda)\,r_\mathrm{P}+\lambda\,r_\mathrm{D}}+r_\mathrm{D}
\frac{dV_\mathrm{D}}{dt}\int_0^1\,\frac{\lambda^2
c_\mathrm{M}(\lambda)\,d\lambda}
{(1-\lambda)\,r_\mathrm{P}+\lambda\,r_\mathrm{D}}\,\Big]\,.
\end{array}
\end{equation}
For tapered segments ($r_\mathrm{P}\ne r_\mathrm{D}$) with
membranes of non-uniform specific capacitance, the integrals in
(\ref{dtc13}) have values
\begin{equation}\label{dtc15}
\begin{array}{rcl}
I^\mathrm{\,cap}_\mathrm{P} & = & \ds 2\pi h
\,r_\mathrm{P}\Big[c_\mathrm{P}\psi(r_\mathrm{P},r_\mathrm{D})
+c_\mathrm{D}\phi(r_\mathrm{P},r_\mathrm{D})\Big]
\frac{dV_\mathrm{P}}{dt}\\[10pt]
&&\qquad\ds+\;2\pi h\Big[c_\mathrm{P}r_\mathrm{D}
\phi(r_\mathrm{P},r_\mathrm{D})+c_\mathrm{D}r_\mathrm{P}
\phi(r_\mathrm{D},r_\mathrm{P})\Big]\frac{dV_\mathrm{D}}{dt}\,,\\[10pt]
-I^\mathrm{\,cap}_\mathrm{D} & = & \ds 2\pi h
\Big[c_\mathrm{P}r_\mathrm{D}\phi(r_\mathrm{P},r_\mathrm{D})
+c_\mathrm{D}r_\mathrm{P}\phi(r_\mathrm{D},r_\mathrm{P})\Big]
\frac{dV_\mathrm{P}}{dt}\\[10pt]
&&\qquad\ds+\;2\pi h r_\mathrm{D}\Big[c_\mathrm{P}
\phi(r_\mathrm{D},r_\mathrm{P})+c_\mathrm{D}\psi(r_\mathrm{D},
r_\mathrm{P})\Big]\frac{dV_\mathrm{D}}{dt}\nonumber
\end{array}
\end{equation}
where $c_\mathrm{M}(\lambda)=(1-\lambda)c_\mathrm{P}+\lambda\,
c_\mathrm{D}$ and the auxiliary functions $\phi(x,y)$ and
$\psi(x,y)$ are defined by
\begin{equation}\label{dtc16}
\begin{array}{rcl}
\phi(x,y) & = & \ds\frac{x}{6(x-y)^3}\,\Big[x^2-5xy-2y^2
+\frac{6xy^2}{x-y}\,\log\frac{x}{y}\,\Big]\,,\\[10pt]
\psi(x,y) & = & \ds\frac{x}{6(x-y)^3}\,\Big[2x^2-7xy+11y^2
-\frac{6y^3}{x-y}\log\frac{x}{y}\,\Big]\,.
\end{array}
\end{equation}
The evaluation of the integrals in expression (\ref{dtc13}) is
facilitated by defining the auxiliary integrals
\[
\mathcal{K}_1= \int_0^1\frac{(1-\lambda)^2
\widehat{c}_\mathrm{M}(\lambda)\,d\lambda}
{\widehat{r}(\lambda)}\,,\qquad \mathcal{K}_2=\int_0^1
\frac{\lambda(1-\lambda)\widehat{c}_\mathrm{M}(\lambda)\,d\lambda}
{\widehat{r}(\lambda)}\,,\qquad
\mathcal{K}_3=\int_0^1\,\frac{\lambda^2\widehat{c}_\mathrm{M}(\lambda)\,d\lambda}
{\widehat{r}(\lambda)}
\]
and observing that $\mathcal{K}_1$, $\mathcal{K}_2$ and
$\mathcal{K}_3$ can be determined easily from the identities
\[
\begin{array}{rcl}
\mathcal{K}_1+2\mathcal{K}_2+\mathcal{K}_3 & = & \ds
\int_0^1\frac{\widehat{c}_\mathrm{M}(\lambda)\,d\lambda}{\widehat{r}(\lambda)}\,,\\[10pt]
r_\mathrm{P}\,\mathcal{K}_1+r_\mathrm{D}\,\mathcal{K}_2 & = &
\ds\int_0^1 (1-\lambda)\widehat{c}_\mathrm{M}(\lambda)\,d\lambda\,,\\[10pt]
r_\mathrm{P}\,\mathcal{K}_2+r_\mathrm{D}\,\mathcal{K}_3 & = & \ds\int_0^1
\lambda\,\widehat{c}_\mathrm{M}(\lambda)\,d\lambda\,.
\end{array}
\]
The results given in subsection \ref{CapCurrent} for a uniform
segment ($r_\mathrm{P}=r_\mathrm{D}$) are obtained from formulae
(\ref{dtc15}) by replacing $\phi(x,y)$ and $\psi(x,y)$ with their
respective limiting values of $1/12$ and $1/4$ where each limit is
taken as $x\to y$.

\pagebreak[4]

\textbf{Appendix 3 -- Partitioning of voltage-dependent current on
tapered cylinders}

The construction of $I^\mathrm{\,cap}_\mathrm{P}$ and
$I^\mathrm{\,cap}_\mathrm{D}$ for a membrane with non-constant
specific capacitance provides the framework for treating intrinsic
voltage-dependent transmembrane current. For tapered segments with
non-constant membrane conductance, the contributions to the
perturbations in the axial current at the proximal and distal
boundaries of the segment are identical to expressions
(\ref{dtc15}) with $c_\mathrm{P}$ replaced by
$g_\mathrm{P}(V_\mathrm{P};\bs{\theta})\,$ and $c_\mathrm{D}$
replaced by $g_\mathrm{D}(V_\mathrm{D};\bs{\theta})\,$. These
contributions are
\begin{equation}\label{dtc19}
\begin{array}{rcl}
I^\mathrm{\,IVDC}_\mathrm{P} & = & 2\pi h
\,r_\mathrm{P}\Big[g_\mathrm{P}(V_\mathrm{P};\bs{\theta})\,\psi(r_\mathrm{P},r_\mathrm{D})
+g_\mathrm{D}(V_\mathrm{D};\bs{\theta})\,\phi(r_\mathrm{P},r_\mathrm{D})\Big]
(V_\mathrm{P}-E)\\[5pt]
&&\qquad+\;2\pi
h\Big[g_\mathrm{P}(V_\mathrm{P};\bs{\theta})\,r_\mathrm{D}
\phi(r_\mathrm{P},r_\mathrm{D})+g_\mathrm{D}(V_\mathrm{D};\bs{\theta})\,r_\mathrm{P}
\phi(r_\mathrm{D},r_\mathrm{P})\Big](V_\mathrm{D}-E)\,,\\[10pt]
-I^\mathrm{\,IVDC}_\mathrm{D} & = & 2\pi h
\Big[g_\mathrm{P}(V_\mathrm{P};\bs{\theta})\,r_\mathrm{D}\phi(r_\mathrm{P},r_\mathrm{D})
+g_\mathrm{D}(V_\mathrm{D};\bs{\theta})\,r_\mathrm{P}\phi(r_\mathrm{D},r_\mathrm{P})\Big]
(V_\mathrm{P}-E)\\[5pt]
&&\qquad+\;2\pi
h\,r_\mathrm{D}\Big[g_\mathrm{P}(V_\mathrm{P};\bs{\theta})\,
\phi(r_\mathrm{D},r_\mathrm{P})+g_\mathrm{D}(V_\mathrm{D};\bs{\theta})\,
\psi(r_\mathrm{D},r_\mathrm{P})\Big](V_\mathrm{D}-E)
\end{array}
\end{equation}
where the auxiliary functions $\phi(x,y)$ and $\psi(x,y)$ are
defined in (\ref{dtc16}).

\textbf{Appendix 4 -- Analytical solution for somal potential of
test neuron}

It may be shown that $V(t)$, the deviation of the somal
transmembrane potential from its resting value as a result of a
distribution $\mathcal{I}(x,t)$ of current on a uniform
cylindrical dendrite of radius $a$ and length $l$ attached to a
soma is
\begin{equation}\label{es4}
V(t)=e^{-t/\tau}\,\Big[\,\phi_0(t)+\sum_\beta\;\phi_\beta(t)
e^{-\beta^2 t/L^2\tau}\,\cos\beta\,\Big]\,, \qquad
L=l\,\sqrt{\ds\frac{2 g_\mathrm{M}}{a g_\mathrm{A}}}
\end{equation}
where $\tau$ is the time constant of the somal and dendritic
membranes and $g_\mathrm{M}$ and $g_\mathrm{A}$ have their usual
meanings. The summation is taken over all the solutions $\beta$ of
the transcendental equation $\tan\beta+\gamma\beta=0$ where
$\gamma$ (constant) is the ratio of the total membrane area of the
soma to the total membrane area of the dendrite. The functions
$\phi_0(t)$ and $\phi_\beta(t)$ are solutions of the differential
equations
\begin{equation}\label{es9}
\begin{array}{rcl}
\ds\frac{d\phi_0}{dt} & = & -\ds\frac{e^{t/\tau}}
{C_\mathrm{D}+C_\mathrm{S}}\,\Big[\,
\mathcal{I}_\mathrm{S}(t)+\int_0^l\,\mathcal{I}(x,t)\,dx\,\Big]\,,\\[10pt]
\ds\frac{d\phi_\beta}{dt} & = &
-\ds\frac{2e^{(1+\beta^2/L^2)t/\tau}} {C_\mathrm{D}
+C_\mathrm{S}\cos^2\beta}\,\Big[\,
\int_0^1\,\mathcal{I}(x,t)\cos\beta\big(1-x/l\big)\,dx
+\cos\beta\,\mathcal{I}_\mathrm{S}(t)\,\Big]
\end{array}
\end{equation}
with initial conditions $\phi_0(0)=\phi_\beta(0)=0$, that is, the
neuron is initialised at its resting potential. The parameters
$C_\mathrm{S}$ and $C_\mathrm{D}$ denote respectively the total
membrane capacitances of the soma and dendrite, and
$\mathcal{I}_\mathrm{S}(t)$ is the current supplied to the soma.

In the special case in which point currents $\mathcal{I}_1(t),
\cdots,\mathcal{I}_n(t)$ act at distances $x_1,\cdots x_n$ from
the soma of the uniform cylinder, the corresponding coefficient
functions $\phi_0$ and $\phi_\beta$ satisfy
\begin{equation}\label{ec2}
\begin{array}{rcl}
\ds\frac{d\phi_0}{dt} & = &\ds -\frac{e^{t/\tau}}
{C_\mathrm{D}+C_\mathrm{S}}\,\Big[\,
\mathcal{I}_\mathrm{S}(t)+\sum_{k=1}^n\;\mathcal{I}_k(t)\,\Big]\,,\\[10pt]
\ds\frac{d\phi_\beta}{dt} & = &
\ds-\frac{2e^{(1+\beta^2/L^2)t/\tau}}
{C_\mathrm{D}+C_\mathrm{S}\cos^2\beta}\,\Big[\, \sum_{k=1}^n
\;\mathcal{I}_k(t)\cos\beta\big(1-x_k/l\big)
+\cos\beta\,\mathcal{I}_\mathrm{S}(t)\,\Big]\,.
\end{array}
\end{equation}
