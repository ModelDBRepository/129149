\section{Results}
The test procedure by which the numerical and analytical solutions
of the model equations are compared will involve various levels of
discretisation of the branched dendrite.

The comparison of the behaviour of the traditional compartmental
model, represented by the NEURON simulator (Hines and Carnevale,
\cite{Hines97}), and the generalised compartmental model is based
on simulations of large scale synaptic activity across the model
dendrite illustrated in Figure \ref{TestNeuron}. Each simulation begins
by distributing nodes across the test dendrite such that each
dendritic section begins and ends on a node, and that within each
dendritic section the nodes are uniformly spaced. For a given
discretisation, 2000 simulations of the traditional and
generalised models under identical conditions are carried out.
Each simulation starts with the dendrite at rest and places a
single exogenous input at a randomly chosen location on the
dendrite. The potential at the soma is computed by solving the
differential equations numerically for each model, and the
magnitude of the relative error in this potential is recorded at
$1$ms intervals up to and including $10$ms after the exogenous
current is applied (see the Appendix for tables of complete
results). The calculation of the relative error is made possible
by the fact that an analytical solution is available for all
distributions of exogenous input on the branched dendrite.

At large (approximate) internodal distances, the discretisation of
each section of the test dendrite may be very different, but as
the maximum internodal distance decreases, the variation in
internodal distances across different dendritic sections
approaches zero. For example, the number of nodes used in the
simulations here will vary from 21 ($166.8\mu\mbox{m}\le h \le
501.2\mu$m) to 992 ($h\approx7.7\mu$m) where $h$ is the internodal
distance. Since the computational effort involved in the numerical
problem is entirely determined by the number of nodes appearing in
the numerical scheme, under all circumstances the same
discretisation is used in the numerical solution of both models.
Moreover, a temporal discretisation of $dt=0.001$ms was used in
all integrations to ensure that errors due to integration were
insignificant compared with those introduced by the spatial
discretisation process.

The results illustrate separately the effect of the two major
assumptions made in the construction of the traditional
compartmental model and relaxed in the construction of the
generalised compartmental model. Recall that in the traditional
model all compartments are assumed to be iso-potential and all
input is assumed to act at the designated node of the compartment.
We use the term \emph{modified compartmental model} to denote a
compartmental model in which the compartments are iso-potential
but the input is distributed in accordance with the scheme set out
for the generalised model. In practice, this means that exogenous
input is partitioned between the nodes either side of its point of
application. The fraction falling on each of these nodes will
depend on the geometry of the dendritic section and the location
of the point of application of the input. The modified
compartmental model in which the iso-potential assumption remains
in place allows one to investigate the effect that placement alone
has on the accuracy the numerical solution. The move to the
generalised model now relaxes the iso-potential assumption of the
modified model and allows the combined effect of both
generalisations to come into play.

Figure \ref{mean} shows the percentage mean value of the modulus
of the relative error of the membrane potential of the soma at
times 2ms and 8ms after the application of a sustained stimulus at
a randomly chosen location on the dendrite. For a given level of
discretisation, identical for each compartmental model, this mean
value is based on 2000 simulations of the effect of the stimulus
which is again placed randomly, but identically, in each model.
Figure \ref{mean} illustrates that for any level of discretisation
the error in the NEURON simulator (Fig. \ref{mean} dotted line) is
significantly greater than that in either the modified model (Fig.
\ref{mean} dashed line) or generalised model (Fig. \ref{mean}
solid line). For example, to achieve a 1\% mean error, the NEURON
simulator needs approximately 400 nodes, the modified model 65
nodes and the generalised model 50 nodes. In this example, it is
the accurate placement of input that has greatest impact on
reducing the error, although the benefit obtained by relaxing the
iso-potential assumption is also significant.

\begin{figure}[!h]
\centering
\begin{tabular}{cc}
\begin{mfpic}[0.4][60]{-30}{400}{-0.2}{2.3}
\headlen7pt
\pen{1pt}
\dotspace=4pt
\dotsize=1.5pt
%
% x-axis
\lines{(0,0),(400,0)}
\lines{(0,0),(0,2)}
\lines{(0,0),(0,-0.1)}
\lines{(100,0),(100,-0.1)}
\lines{(200,0),(200,-0.1)}
\lines{(300,0),(300,-0.1)}
\lines{(400,0),(400,-0.1)}
\tlabel[tc](0,-0.2){\textsf{0}}
\tlabel[tc](100,-0.2){\textsf{100}}
\tlabel[tc](200,-0.2){\textsf{200}}
\tlabel[tc](300,-0.2){\textsf{300}}
\tlabel[tc](400,-0.2){\textsf{400}}
%
% y-axis
\lines{(0,0.0),(-10,0.0)}
\lines{(0,0.5),(-10,0.5)}
\lines{(0,1.0),(-10,1.0)}
\lines{(0,1.5),(-10,1.5)}
\lines{(0,2.0),(-10,2.0)}
\tlabel[cr](-15,0.0){\textsf{0.0}}
\tlabel[cr](-15,0.5){\textsf{0.5}}
\tlabel[cr](-15,1.0){\textsf{1.0}}
\tlabel[cr](-15,1.5){\textsf{1.5}}
\tlabel[cr](-15,2.0){\textsf{2.0}}
\tlabel[cr](350,1.75){\textsf{t = 2ms}}
%
% Gen Mean error at t=2
\curve{ ( 34,1.807),( 43,1.114),( 54,0.632),( 61,0.514),(
75,0.338), ( 82,0.269),(
93,0.211),(119,0.132),(142,0.091),(169,0.060),
(193,0.050),(244,0.029),(293,0.021),(391,0.011)}
%
% Mod Mean error at t=2
\dashed\curve{ ( 43, 2.226),( 54, 1.258),( 61, 1.049),( 75,
0.680),( 82, 0.550), ( 93, 0.447),(119, 0.273),(142, 0.181),(169,
0.128),(193, 0.101), (244, 0.063),(293, 0.043),(391, 0.023)}
%
% Old Mean error at t=2
\dotted\curve{ (193, 2.072),(244, 1.657),(293, 1.397),(391,
1.005)}
\end{mfpic} &\qquad
\begin{mfpic}[0.4][60]{-30}{400}{-0.2}{2.3}
\headlen7pt
\pen{1pt}
\dotspace=4pt
\dotsize=1.5pt
%
% x-axis
\lines{(0,0),(400,0)} \lines{(0,0),(0,2)} \lines{(0,0),(0,-0.1)}
\lines{(100,0),(100,-0.1)} \lines{(200,0),(200,-0.1)}
\lines{(300,0),(300,-0.1)} \lines{(400,0),(400,-0.1)}
\tlabel[tc](0,-0.2){\textsf{0}}
\tlabel[tc](100,-0.2){\textsf{100}}
\tlabel[tc](200,-0.2){\textsf{200}}
\tlabel[tc](300,-0.2){\textsf{300}}
\tlabel[tc](400,-0.2){\textsf{400}}
%
% y-axis
\lines{(0,0.0),(-10,0.0)} \lines{(0,0.5),(-10,0.5)}
\lines{(0,1.0),(-10,1.0)} \lines{(0,1.5),(-10,1.5)}
\lines{(0,2.0),(-10,2.0)} \tlabel[cr](-15,0.0){\textsf{0.0}}
\tlabel[cr](-15,0.5){\textsf{0.5}}
\tlabel[cr](-15,1.0){\textsf{1.0}}
\tlabel[cr](-15,1.5){\textsf{1.5}}
\tlabel[cr](-15,2.0){\textsf{2.0}}
\tlabel[cr](350,1.75){\textsf{t = 8ms}}
%
% Gen Mean error at t=8
\curve{ ( 21,1.732),( 34,0.412),( 43,0.252),( 54,0.140),(
61,0.119), ( 75,0.077),( 82,0.062),(
93,0.050),(119,0.031),(142,0.020),
(169,0.014),(193,0.011),(244,0.007),(293,0.004),(391,0.002)}

%
% Mod Mean error at t=8
\dashed\curve{ ( 21,1.941),( 34,0.477),( 43,0.294),( 54,0.163),(
61,0.142), ( 75,0.090),( 82,0.073),(
93,0.059),(119,0.037),(142,0.024),
(169,0.016),(193,0.013),(244,0.008),(293,0.005),(391,0.003)}
%
% Old Mean error at t=8
\dotted\curve{ ( 54,2.155),( 61,1.929),( 75,1.552),( 82,1.404),(
93,1.262),
(119,1.000),(142,0.780),(169,0.665),(193,0.586),(244,0.469),
(293,0.396),(391,0.286)}
\end{mfpic}
\end{tabular}
\parbox{6in}{\caption{\label{mean} The percentage mean value of
the modulus of the relative error of the membrane potential of the
soma at 2ms and 8ms after the application of a randomly placed
sustained stimulus is plotted against the number of nodes used to
discretise the dendrite for the NEURON simulator (dotted line),
the modified compartmental model (dashed line) and the generalised
compartmental model (solid line).}}
\end{figure}

The conclusion from Figure \ref{mean} is that the generalised
compartmental model performs better than both the modified and
traditional models at all levels of discretisation. Therefore the
ratio of the errors in the modified and NEURON simulator models to
the error in the generalised model is a sensible measure of
relative performance. The Common logarithm of this ratio is most
useful for representational purposes as it will always be positive
since the generalised model is known \emph{a priori} to perform
better than either of the other two models. Figure \ref{sdev}
indicates that this ratio is relatively insensitive to time for
all levels of discretisation. This observation suggests that
although the relative error in all models decreases with time
(Figure \ref{mean}), the rate at which this occurs is effectively
the same for all three models. More importantly, Figure \ref{sdev}
also shows that the accuracy of the NEURON simulator (dashed line)
relative to that of the generalised model deteriorates as spatial
resolution is refined. The inference one would draw from this
observation is that the NEURON simulator and the generalised
compartmental model have different orders of numerical accuracy
with respect to spatial discretisation. On the other hand, the
modified model (solid line) and the generalised model appear to
enjoy the same level of numerical accuracy with respect to spatial
discretisation.

An examination of the standard deviations of the errors (not
shown) in the three models leads to a similar set of conclusions
as those described for the relative accuracy of the different
models. The standard deviation of the error in the generalised
model is again the smallest of the three models, and so the Common
logarithm of the ratios of the standard deviations in the modified
and NEURON simulator models to that in the generalised model is a
good measure. The Common logarithm of the two ratios is again
largely independent of time for both models. With respect to
spatial discretisation, this ratio is independent of the level of
spatial disctetisation  for all practical purposes. On the other
hand, the same ratio for the NEURON simulator grows in a similar
way to that shown in Figure \ref{mean} (dashed lines) as the
spatial discretisation is refined. So not only is the accuracy
achieved by the NEURON simulator significantly less than that of
the generalised model, it is also more erratic and therefore less
precise.

\begin{figure}[!h]
\[
\begin{array}{c}
$\begin{mfpic}[15][60]{-1}{12}{-0.3}{2.7}
\headlen7pt
\pen{1pt}
\dotspace=4pt
\dotsize=1.5pt
%
% x-axis
\lines{(0,0),(0,2.5)}
\lines{(0,0),(10,0)}
\lines{(0,0),(0,-0.1)}
\lines{(2,0),(2,-0.1)}
\lines{(4,0),(4,-0.1)}
\lines{(6,0),(6,-0.1)}
\lines{(8,0),(8,-0.1)}
\lines{(10,0),(10,-0.1)}
\tlabel[tc](0,-0.2){\textsf{0.0}}
\tlabel[tc](2,-0.2){\textsf{2.0}}
\tlabel[tc](4,-0.2){\textsf{4.0}}
\tlabel[tc](6,-0.2){\textsf{6.0}}
\tlabel[tc](8,-0.2){\textsf{8.0}}
\tlabel[tc](10,-0.2){\textsf{10.0}}
%
% y-axis
\lines{(0,0.0),(-0.3,0.0)}
\lines{(0,0.5),(-0.3,0.5)}
\lines{(0,1.0),(-0.3,1.0)}
\lines{(0,1.5),(-0.3,1.5)}
\lines{(0,2.0),(-0.3,2.0)}
\lines{(0,2.5),(-0.3,2.5)}
\tlabel[cr](-0.5,0.0){\textsf{0.0}}
\tlabel[cr](-0.5,0.5){\textsf{0.5}}
\tlabel[cr](-0.5,1.0){\textsf{1.0}}
\tlabel[cr](-0.5,1.5){\textsf{1.5}}
\tlabel[cr](-0.5,2.0){\textsf{2.0}}
\tlabel[cr](-0.5,2.5){\textsf{2.5}}
%
% Mod Model at 21 nodes
\lines{ ( 1.0,0.319),( 2.0,0.214),( 3.0,0.140),( 4.0,0.098),
        ( 5.0,0.075),( 6.0,0.061),( 7.0,0.054),( 8.0,0.049),
        ( 9.0,0.047),(10.0,0.045)}
\tlabel[cr](0.8,0.319){\textsf{21}}
%
% Old Model at 21 nodes
\dashed\lines{ ( 1.0,0.285),( 2.0,0.430),( 3.0,0.492),( 4.0,0.523),
               ( 5.0,0.537),( 6.0,0.544),( 7.0,0.549),( 8.0,0.551),
               ( 9.0,0.554),(10.0,0.555)}
\tlabel[cl](10.5,0.555){\textsf{21}}
%
% Mod Model at 54 nodes
\lines{ ( 1.0,0.487),( 2.0,0.299),( 3.0,0.188),( 4.0,0.128),(
5.0,0.097), ( 6.0,0.080),( 7.0,0.071),( 8.0,0.067),(
9.0,0.064),(10.0,0.064)}
%
% Old Model at 54 nodes
\dashed\lines{ ( 1.0,0.898),( 2.0,1.071),( 3.0,1.127),( 4.0,1.151),
               ( 5.0,1.165),( 6.0,1.175),( 7.0,1.181),( 8.0,1.186),
               ( 9.0,1.190),(10.0,1.191)}
\tlabel[cl](10.5,1.191){\textsf{54}}
%
% Mod Model at 119 nodes
\lines{ ( 1.0,0.529),( 2.0,0.315),( 3.0,0.195),( 4.0,0.136),
        ( 5.0,0.104),( 6.0,0.087),( 7.0,0.078),( 8.0,0.072),
        ( 9.0,0.069),(10.0,0.069)}
%
% Old Model at 119 nodes
\dashed\lines{ ( 1.0,1.268),( 2.0,1.422),( 3.0,1.459),( 4.0,1.477),
               ( 5.0,1.487),( 6.0,1.494),( 7.0,1.499),( 8.0,1.503),
               ( 9.0,1.506),(10.0,1.503)}
\tlabel[cl](10.5,1.503){\textsf{119}}
%
% Mod Model at 244 nodes
\lines{ ( 1.0,0.536),( 2.0,0.326),( 3.0,0.205),( 4.0,0.142),
        ( 5.0,0.108),( 6.0,0.090),( 7.0,0.080),( 8.0,0.074),
        ( 9.0,0.071),(10.0,0.074)}
%
% Old Model at 244 nodes
\dashed\lines{ ( 1.0,1.581),( 2.0,1.744),( 3.0,1.782),( 4.0,1.798),
               ( 5.0,1.807),( 6.0,1.814),( 7.0,1.818),( 8.0,1.822),
               ( 9.0,1.826),(10.0,1.821)}
\tlabel[cl](10.5,1.821){\textsf{244}}
%
% Mod Model at 495 nodes
\lines{ ( 1.0,0.514),( 2.0,0.309),( 3.0,0.194),( 4.0,0.135),
        ( 5.0,0.103),( 6.0,0.086),( 7.0,0.077),( 8.0,0.072),
        ( 9.0,0.069),(10.0,0.071)}
%
% Old Model at 495 nodes
\dashed\lines{ ( 1.0,1.873),( 2.0,2.045),( 3.0,2.091),( 4.0,2.111),
               ( 5.0,2.122), ( 6.0,2.130),( 7.0,2.136),( 8.0,2.140),
               ( 9.0,2.144),(10.0,2.138)}
\tlabel[cl](10.5,2.138){\textsf{495}}
%
% Mod Model at 992 nodes
\lines{ ( 1.0,0.510),( 2.0,0.312),( 3.0,0.197),( 4.0,0.136),
        ( 5.0,0.106),( 6.0,0.089),( 7.0,0.079),( 8.0,0.073),
        ( 9.0,0.072),(10.0,0.062)}
%
% Old Model at 992 nodes
\dashed\lines{ ( 1.0,2.168),( 2.0,2.352),( 3.0,2.399),( 4.0,2.418),
               ( 5.0,2.430),( 6.0,2.439),( 7.0,2.444),( 8.0,2.447),
               ( 9.0,2.453),(10.0,2.414)}
\tlabel[cl](10.5,2.414){\textsf{992}}
\end{mfpic}$
\end{array}\qquad
\begin{tabular}{p{2.5in}}
\caption{\label{sdev} The Common logarithm of
the ratio of the mean value of the modulus of the relative error
in the NEURON simulator to that of the generalised compartmental
model (dashed line) and the modified compartmental model to that
of the generalised compartmental model (solid line) are plotted
against time for various levels of spatial discretisation.}
\end{tabular}
\]
\end{figure}

\section{Concluding remarks}
This investigation has demonstrated that it is possible to achieve
a cost free increase in the accuracy and precision of
compartmental models once the assumption of iso-potential
compartments is relaxed and the actual placement of input taken
into account. In the case of large scale exogenous input, it was
clear that placement was the major determinant of accuracy but
that the relaxation of the iso-potential assumption produced a
further significant reduction in overall error. One would
anticipate that the physiological manifestations of these
differences in accuracy would appear as errors in the measurement
of the time constant of the somal membrane in response to a
sustained input on a dendrite.

\section*{Acknowledgement}
Alan E Lindsay would like to thank the Wellcome Trust for the
award of Vacation Scholarship (VS/03/GLA/8/SL/TH/FH) which was
used to fund this work.
