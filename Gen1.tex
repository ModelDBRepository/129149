\section{Introduction}
The traditional compartmental model of a neuron, in which the
behaviour of its complexly branched dendrites is described by the
solution of a family of ordinary differential equations, was
originally developed as a means of reducing the mathematical
complexity associated with the continuum description of a neuron
in terms of a family of connected partial differential equations
(Rall,\cite{Rall64}). The question posed in this project is
whether or not the accuracy of the traditional compartmental model
of a dendrite can be significantly improved without additional
computational effort. Toward this end, a generalised compartmental
model is developed and its computational properties examined with
reference to the behaviour of a test neuron with known exact
solution for given input. To facilitate this development, the
terminology used to describe the morphology of a neuron and its
input/output structure is now set out.

\begin{figure}[!h]
%THE FULLY IDEALISED NEURON
\centerline{\begin{mfpic}[0.8][0.8]{10}{210}{-80}{270}
\pen{1pt}
% Draws the dendrite
\shade\cyclic{(80,14),(85,15),(93,10),(100,7),(110,5),(110,3),(100,-3),(90,-5),(80,14)}
\curve{(80,14),(85,15),(93,10),(100,7),(110,5),(110,3),(100,-3),(90,-5),(80,14)}
\gfill\cyclic{(90,-5),(100,-2),(90,3),(85,1),(90,-5)}
\curve{(110,4),(115,10),(120,15),(125,19),(130,22),
(135,24),(140,25),(145,24),(150,22),(155,19)}
\curve{(135,24),(140,27),(145,32),(150,39),(155,48)}
\curve{(125,19),(120,30),(123,50),(125,80),(128,100),(126,120),(124,140),(128,160),(136,180),(148,200)}
\curve{(125,80),(130,85),(135,95),(140,110),(145,130)}
\curve{(124,140),(122,150),(120,158),(118,164)}
\curve{(80,14),(75,15),(70,17),(65,19.5),(60,23),(55,30)}
\curve{(80,14),(85,30),(90,50),(80,70),(70,80),(60,90),(50,110),(40,140),(5,190)}
\curve{(50,110),(60,125),(70,160),(80,195),(115,235)}
\curve{(80,70),(85,80),(90,100),(95,140)}
\curve{(85,30),(75,35),(65,45),(50,50),(40,65)}
% Draw axon
\pen{2pt}
\curve{(90,-5),(95,-20),(93,-30),(92,-40)}
\pen{1pt}
\lines{(118,-7),(118,-10),(143,-10),(143,-7)}
\tlabel[cc](130,-18){\footnotesize $100\;\mu$m}
\headlen7pt
% Deals with nodes
\tlabel[bc](180,170){\footnotesize branch}
\tlabel[bc](180,160){\footnotesize point}
\lines{(180,155),(165,140)}
\arrow\lines{(165,140),(129,140)}
\tlabel[bc](180,110){\footnotesize branch}
\tlabel[bc](180,100){\footnotesize point}
\lines{(180,95),(165,80)}
\arrow\lines{(165,80),(130,80)}
% Deals with terminal
\tlabel[bc](190,215){\footnotesize terminals}
\lines{(190,210),(180,200)}
\lines{(190,225),(180,235)}
\arrow\lines{(180,235),(120,235)}
\arrow\lines{(180,200),(153,200)}
% Deals with segments
\tlabel[bc](190,25){\footnotesize sections}
\lines{(190,35),(170,55)}
\arrow\lines{(170,55),(130,55)}
\lines{(190,20),(180,10)}
\arrow\lines{(180,10),(120,10)}
% Deals with Soma
\tlabel[bc](25,15){\footnotesize Soma showing}
\tlabel[bc](25,5){\footnotesize trigger zone}
\lines{(25,0),(30,-5)}
\arrow\lines{(30,-5),(80,-5)}
% Deals with Axon
\tlabel[cr](60,-25){\footnotesize Axon}
\arrow\lines{(65,-25),(90,-25)}
% Somal spike train
\curve{(30,-70),(39.2,-70),(40,-68),(40.2,-66),(40.4,-64),
(40.6,-62),(40.8,-60),(41,-58),(41.2,-56),(41.4,-54),
(41.6,-52),(41.8,-50),(42,-49),(43,-60),(44,-71),(45,-70)}
\lines{(45,-70),(49.2,-70)}
\curve{(49.2,-70),(50,-68),(50.2,-66),(50.4,-64),(50.6,-62),(50.8,-60),
(51,-58),(51.2,-56),(51.4,-54),(51.6,-52), (51.8,-50),(52,-49),(53,-60),
(54,-71),(55,-70)}
\lines{(55,-70),(69.2,-70)}
\curve{(69.2,-70),(70,-68),(70.2,-66),(70.4,-64),(70.6,-62),(70.8,-60),
(71,-58),(71.2,-56),(71.4,-54),(71.6,-52), (71.8,-50),(72,-49),(73,-60),
(74,-71),(75,-70)}
\lines{(75,-70),(83.2,-70)}
\curve{(83.2,-70),(84,-68),(84.2,-66),(84.4,-64),(84.6,-62),(84.8,-60),
(85,-58),(85.2,-56),(85.4,-54),(85.6,-52), (85.8,-50),(85,-49),(86,-60),
(87,-71),(88,-70)}
\lines{(88,-70),(114.2,-70)}
\curve{(114.2,-70),(115,-68),(115.2,-66),(115.4,-64),(115.6,-62),
(115.8,-60),(116,-58),(116.2,-56),(116.4,-54),(116.6,-52),
(116.8,-50),(117,-49),(118,-60),(119,-71),(120,-70)}
\lines{(120,-70),(130,-70)}
% Input spike train
\curve{(10,250),(19.2,250),(20,252),(20.2,254),(20.4,256),(20.6,258),
(20.8,260),(21,262),(21.2,264),(21.4,266),(21.6,268), (21.8,270),
(22,271),(23,260),(24,249),(25,250)}
\curve{(25,250),(29.2,250),(30,252),(30.2,254),(30.4,256),(30.6,258),
(30.8,260),(31,262),(31.2,264),(31.4,266),(31.6,268),(31.8,270),
(32,271),(33,260),(34,249),(35,250)}
\curve{(35,250),(39.2,250),(40,252),(40.2,254),(40.4,256),(40.6,258),
(40.8,260),(41,262),(41.2,264),(41.4,266),(41.6,268), (41.8,270),(42,271),(43,260),(44,249),(45,250)}
\lines{(45,250),(49.2,250)}
\curve{(49.2,250),(50,252),(50.2,254),(50.4,256),(50.6,258),
(50.8,260),(51,262),(51.2,264),(51.4,266),(51.6,268), (51.8,270),(52,271),(53,260),(54,249),(55,250)}
\lines{(55,250),(69.2,250)}
\curve{(69.2,250),(70,252),(70.2,254),(70.4,256),(70.6,258),
(70.8,260),(71,262),(71.2,264),(71.4,266),(71.6,268), (71.8,270),(72,271),(73,260),(74,249),(75,250)}
\lines{(75,250),(83.2,250)}
\curve{(83.2,250),(84,252),(84.2,254),(84.4,256),(84.6,258),
(84.8,260),(85,262),(85.2,264),(85.4,266),(85.6,268), (85.8,270),(85,271),(86,260),(87,249),(88,250)}
\lines{(88,250),(99.2,250)}
\curve{(99.2,250),(100,252),(100.2,254),(100.4,256),(100.6,258),
(100.8,260),(101,262),(101.2,264),(101.4,266),(101.6,268),
(101.8,270),(102,271),(103,260),(104,249),(105,250)}
\lines{(105,250),(114.2,250)}
\curve{(114.2,250),(115,252),(115.2,254),(115.4,256),(115.6,258),
(115.8,260),(116,262),(116.2,264),(116.4,266),(116.6,268),
(116.8,270),(117,271),(118,260),(119,249),(120,250)}
\lines{(120,250),(130,250)}
\tlabel[cc](180,265){\footnotesize An Input}
\tlabel[cc](180,255){\footnotesize Spike Train}
\arrow\lines{(145,260),(130,260)}
%  Heading
\tlabel[bc](180,-30){\footnotesize Output Spike}
\tlabel[bc](180,-40){\footnotesize Train}
\lines{(180,-45),(165,-60)}
\arrow\lines{(165,-60),(130,-60)}
% Synaptic inputs
\pen{0.5pt}
\curve{(76,199),(75,200),(74,201),(73,203),(72,207),(71,213),(70,223),(69,240)}
\lines{(77,194),(76,199),(80,200)}
\curve{(91,220),(90,222),(89,223),(86,225),(81,226),(73,225),(69,225)}
\lines{(91,215),(91,220),(96,220)}
\lines{(17,180),(23,180),(23,174)}
\curve{(23,180),(30,183),(40,186),(50,188),(60,189),(65,190),(66,191),(67,193),
(68,196),(69,201),(70,209),(71,213)}
\lines{(37,153),(41,151),(41,145)}
\curve{(41,151),(45,155),(50,160),(55,166),(61,174),(64,185),(65,190)}
\tlabel[cc](30,235){\footnotesize Synaptic}
\tlabel[cc](30,225){\footnotesize inputs}
\arrow\lines{(30,215),(50,195)}
% Deal with tree heading
\tlabel[bc](20,110){\footnotesize An}
\tlabel[bc](20,100){\footnotesize Idealised}
\tlabel[bc](20,90){\footnotesize Dendritic}
\tlabel[bc](20,80){\footnotesize  Tree}
\end{mfpic}}
\centering
\parbox{2.2in}{\caption{\label{neuron} A stylised neuron with
some common nomenclature.}}
\end{figure}

Figure \ref{neuron} illustrates a typical neuron with two
dendrites or branching structures emanating from the cell body
(soma) of the neuron. The length of dendrite connecting one branch
point to a neighbouring branch point, the soma or a dendritic
terminal is called a dendritic section. In practice, dendritic
sections are divided into shorter units called segments. Following
Segev and Burke (\cite{Segev98}), a segment is assumed to be a
uniform cylinder and a section is represented by a series of such
cylinders as illustrated in Figure \ref{ld}.

\pagebreak[4]

\begin{figure}[!h]
\centerline{\begin{mfpic}[0.8][1]{0}{350}{160}{320}
\headlen7pt
\pen{1pt}
\dotspace=4pt
\dotsize=1.5pt
%
% Partial cylinder on right
\ellipse{(350,240),12,16}
\lines{(350,256),(380,256)}
\dashed\lines{(380,256),(400,256)}
\lines{(350,224),(380,224)}
\dashed\lines{(380,224),(400,224)}
%
% RH cylinder
\ellipse{(350,240),18,24}
\lines{(250,264),(350,264)}
\lines{(250,216),(350,216)}
\dotted\parafcn[s]{0,180,5}{(250+18*sind(t),240+24*cosd(t))}
\parafcn[s]{0,180,5}{(250-18*sind(t),240+24*cosd(t))}
%
% Central cylinder
\dotted\parafcn[s]{0,180,5}{(250+24*sind(t),240+32*cosd(t))}
\parafcn[s]{-30,210,5}{(250-24*sind(t),240+32*cosd(t))}
\lines{(100,272),(250,272)}
\lines{(100,208),(250,208)}
\dotted\parafcn[s]{0,180,5}{(100+24*sind(t),240+32*cosd(t))}
\parafcn[s]{0,180,5}{(100-24*sind(t),240+32*cosd(t))}
%
% LH cylinder
\dotted\parafcn[s]{0,180,5}{(100+30*sind(t),240+40*cosd(t))}
\parafcn[s]{-30,210,5}{(100-30*sind(t),240+40*cosd(t))}
\lines{(0,280),(100,280)}
\lines{(0,200),(100,200)}
\dotted\parafcn[s]{0,180,5}{(30*sind(t),240+40*cosd(t))}
\parafcn[s]{0,180,5}{(-30*sind(t),240+40*cosd(t))}
%
% Partial cylinder on left
\dotted\parafcn[s]{0,180,5}{(36*sind(t),240+48*cosd(t))}
\parafcn[s]{-30,210,5}{(-36*sind(t),240+48*cosd(t))}
\lines{(-30,288),(0,288)}
\lines{(-30,192),(0,192)}
\dashed\lines{(-50,288),(-30,288)}
\dashed\lines{(-50,192),(-30,192)}
%
% Delimit cylinders
\arrow\lines{(-50,300),(-5,300)}
\arrow\lines{(50,300),(5,300)}
\arrow\lines{(50,300),(95,300)}
\arrow\lines{(175,300),(105,300)}
\arrow\lines{(175,300),(245,300)}
\arrow\lines{(300,300),(255,300)}
\arrow\lines{(300,300),(345,300)}
\arrow\lines{(400,300),(355,300)}
%
% Annotation of LH cylinder
\arrow\lines{(0,240),(20,240)}
\arrow\lines{(100,240),(150,240)}
\tlabel[br](112,250){$I_\mathrm{LC}$}
\tlabel[bc](50,250){$V_\mathrm{L}$}
\tlabel[tl](55,235){\large $x_\mathrm{L}$}
\tlabel[cc](50,240){\large $\bullet$}
\arrow\lines{(50,225),(50,180)}
\tlabel[tc](50,170){$I_\mathrm{L}$}
%
% Annotation of Central cylinder
\arrow\lines{(250,240),(290,240)}
\tlabel[br](265,250){$I_\mathrm{CR}$}
\tlabel[bc](175,250){$V_\mathrm{C}$}
\tlabel[tl](180,235){\large $x_\mathrm{C}$}
\tlabel[cc](175,240){\large $\bullet$}
\arrow\lines{(175,225),(175,180)}
\tlabel[tc](175,170){$I_\mathrm{C}$}
%
% Annotation of RH cylinder
\arrow\lines{(350,240),(390,240)}
\tlabel[bc](300,250){$V_\mathrm{R}$}
\tlabel[tl](305,235){\large $x_\mathrm{R}$}
\tlabel[cc](300,240){\large $\bullet$}
\arrow\lines{(300,225),(300,180)}
\tlabel[tc](300,170){$I_\mathrm{R}$}
\end{mfpic}}
\centering
\parbox{5.5in}{\caption{\label{ld}
Three consecutive segments of a dendritic section are illustrated.
Current $I_\mathrm{C}$ flows across the membrane at
$x_\mathrm{C}$, axial currents $I_\mathrm{LC}$ and $I_\mathrm{CR}$
flow from $x_\mathrm{L}$ to $x_\mathrm{C}$ and from $x_\mathrm{C}$
to $x_\mathrm{R}$ respectively through a resistive dendritic
core.}}
\end{figure}

In the traditional compartmental model, each segment is
represented by an elemental circuit with electrical properties
that incorporate the local biophysical and morphological
properties of the dendritic segment it represents. For example,
Figure \ref{circuit} illustrates how the three segments of Figure
\ref{ld} might be represented by three elemental circuits.

\begin{figure}[!h]
\centerline{\begin{mfpic}[1][0.8]{0}{400}{20}{260}
\headlen7pt
\pen{1pt}
% Build axonal resistances
\dashed\lines{(0,230),(27.5,230)}
\lines{(27.5,230),(30,235),(35,225),(40,235),(45,225),
(50,235),(55,225),(60,235),(65,225),(67.5,230),(80,230)}
\lines{(80,230),(92.5,230),(95,235),(100,225),(105,235),
(110,225),(115,235),(120,225),(125,235),(130,225),(135,235),
(140,225),(145,235),(150,225),(155,235),(160,225),(165,235),
(170,225),(175,235),(180,225),(185,235),(187.5,230),(200,230)}
\tlabel[tc](145,220){$R_\mathrm{LC}$}
\lines{(200,230),(212.5,230),(215,235),(220,225),(225,235),
(230,225),(235,235),(240,225),(245,235),(250,225),(255,235),
(260,225),(265,235),(270,225),(275,235),(280,225),(285,235),
(290,225),(295,235),(300,225),(305,235),(307.5,230),(320,230)}
\tlabel[tc](260,220){$R_\mathrm{CR}$}
\lines{(320,230),(332.5,230),(335,235),(340,225),(345,235),
(350,225),(355,235),(360,225),(365,235),(370,225),(372.5,230)}
\dashed\lines{(372.5,230),(400,230)}
\tlabel[cc](80,230){$\bullet$}
\tlabel[cc](200,230){$\bullet$}
\tlabel[cc](320,230){$\bullet$}
\arrow\lines{(100,245),(180,245)}
\arrow\lines{(220,245),(300,245)}
\tlabel[bc](140,255){$I_\mathrm{LC}$}
\tlabel[bc](260,255){$I_\mathrm{CR}$}
\tlabel[bc](80,255){$V_\mathrm{L}$}
\tlabel[bc](200,255){$V_\mathrm{C}$}
\tlabel[bc](320,255){$V_\mathrm{R}$}
% Build Right hand circuit
\lines{(320,230),(320,180),(300,180)}
\lines{(300,142.5),(295,140),(305,135),(295,130),(305,125),
(295,120),(305,115),(295,110),(305,105),(295,100),
(300,97.5),(300,90),(340,90),(340,130)}
\lines{(350,130),(330,130)}
\lines{(350,140),(330,140)}
\tlabel[bl](345,150){$c^{(m)}_\mathrm{R}$}
\lines{(340,140),(340,180),(320,180)}
\lines{(320,90),(320,40)}
\lines{(310,40),(330,40)}
\lines{(312.5,35),(327.5,35)}
\lines{(315,30),(325,30)}
\lines{(317.5,25),(322.5,25)}
\dashed\lines{(300,180),(260,180)}
\dashed\lines{(300,90),(260,90)}
\lines{(300,180),(320,180)}
\lines{(300,180),(300,170)}
% Battery
\lines{(295,170),(305,170)}
\lines{(295,162),(305,162)}
\lines{(295,154),(305,154)}
\lines{(255,150),(265,150)}
\lines{(255,166),(265,166)}
\lines{(255,158),(265,158)}
\pen{2pt}
\lines{(297.5,150),(302.5,150)}
\lines{(297.5,166),(302.5,166)}
\lines{(297.5,158),(302.5,158)}
\lines{(257.5,170),(262.5,170)}
\lines{(257.5,162),(262.5,162)}
\lines{(257.5,154),(262.5,154)}
\pen{1pt}
\lines{(300,150),(300,142.5)}
\lines{(260,180),(260,170)}
\lines{(260,150),(260,142.5),(255,140),(265,135),(255,130),
(265,125),(255,120),(265,115),(255,110),(265,105),(255,100),
(260,97.5),(260,90)}
\arrow\lines{(290,135),(290,105)}
\arrow\lines{(310,80),(310,50)}
\tlabel[cr](305,65){$I^{(m)}_\mathrm{R}$}
% Build Left hand circuit
\lines{(80,230),(80,180),(60,180)}
\lines{(60,142.5),(55,140),(65,135),(55,130),(65,125),
(55,120),(65,115),(55,110),(65,105),(55,100),(60,97.5),
(60,90),(100,90),(100,130)}
\lines{(110,130),(90,130)}
\lines{(110,140),(90,140)}
\tlabel[bl](105,150){$c^{(m)}_\mathrm{L}$}
\lines{(100,140),(100,180),(80,180)}
\lines{(80,90),(80,40)}
\lines{(70,40),(90,40)}
\lines{(72.5,35),(87.5,35)}
\lines{(75,30),(85,30)}
\lines{(77.5,25),(82.5,25)}
\dashed\lines{(60,180),(20,180)}
\dashed\lines{(60,90),(20,90)}
\lines{(60,180),(80,180)}
\lines{(60,180),(60,170)}
% Battery
\lines{(55,170),(65,170)}
\lines{(55,162),(65,162)}
\lines{(55,154),(65,154)}
\lines{(15,150),(25,150)}
\lines{(15,166),(25,166)}
\lines{(15,158),(25,158)}
\pen{2pt}
\lines{(57.5,150),(62.5,150)}
\lines{(57.5,166),(62.5,166)}
\lines{(57.5,158),(62.5,158)}
\lines{(17.5,170),(22.5,170)}
\lines{(17.5,162),(22.5,162)}
\lines{(17.5,154),(22.5,154)}
\pen{1pt}
\lines{(60,150),(60,142.5)}
\lines{(20,180),(20,170)}
\lines{(20,150),(20,142.5),(15,140),(25,135),(15,130),
(25,125),(15,120),(25,115),(15,110),(25,105),(15,100),
(20,97.5),(20,90)}
\arrow\lines{(50,135),(50,105)}
\arrow\lines{(70,80),(70,50)}
\tlabel[cr](65,65){$I^{(m)}_\mathrm{L}$}
% Build Middle circuit
\lines{(200,230),(200,180),(180,180)}
\lines{(180,142.5),(175,140),(185,135),(175,130),(185,125),
(175,120),(185,115),(175,110),(185,105),(175,100),
(180,97.5),(180,90),(220,90),(220,130)}
\lines{(230,130),(210,130)}
\lines{(230,140),(210,140)}
\tlabel[bl](225,150){$c^{(m)}_\mathrm{C}$}
\lines{(220,140),(220,180),(200,180)}
\lines{(200,90),(200,40)}
\lines{(190,40),(210,40)}
\lines{(192.5,35),(207.5,35)}
\lines{(195,30),(205,30)}
\lines{(197.5,25),(202.5,25)}
\dashed\lines{(180,180),(140,180)}
\dashed\lines{(180,90),(140,90)}
\lines{(180,180),(200,180)}
\lines{(180,180),(180,170)}
% Battery
\lines{(175,170),(185,170)}
\lines{(175,162),(185,162)}
\lines{(175,154),(185,154)}
\lines{(135,150),(145,150)}
\lines{(135,166),(145,166)}
\lines{(135,158),(145,158)}
\pen{2pt}
\lines{(177.5,150),(182.5,150)}
\lines{(177.5,166),(182.5,166)}
\lines{(177.5,158),(182.5,158)}
\lines{(137.5,170),(142.5,170)}
\lines{(137.5,162),(142.5,162)}
\lines{(137.5,154),(142.5,154)}
\pen{1pt}
\lines{(180,150),(180,142.5)}
\lines{(140,180),(140,170)}
\lines{(140,150),(140,142.5),(135,140),(145,135),(135,130),
(145,125),(135,120),(145,115),(135,110),(145,105),(135,100),
(140,97.5),(140,90)}
\arrow\lines{(170,135),(170,105)}
\arrow\lines{(190,80),(190,50)}
\tlabel[cr](185,65){$I^{(m)}_\mathrm{C}$}
% Far right
\dashed\lines{(400,180),(380,180)}
\dashed\lines{(400,90),(380,90)}
\lines{(380,150),(380,142.5),(375,140),(385,135),(375,130),
(385,125),(375,120),(385,115),(375,110),(385,105),(375,100),
(380,97.5),(380,90)}
\arrow\lines{(50,135),(50,105)}
\arrow\lines{(70,80),(70,50)}
% Battery
\lines{(375,150),(385,150)}
\lines{(375,166),(385,166)}
\lines{(375,158),(385,158)}
\lines{(380,180),(380,170)}
\pen{2pt}
\lines{(377.5,170),(382.5,170)}
\lines{(377.5,162),(382.5,162)}
\lines{(377.5,154),(382.5,154)}
\pen{1pt}
\end{mfpic}}
\centering
\parbox{5.6in}{\caption{\label{circuit} A diagrammatic
representation of two elemental circuits used to construct the
compartmental model of a dendritic section. Axial current
$I_\mathrm{L}$ flows in the left hand compartment under the
influence of the potential difference
$(V_\mathrm{L}-V_\mathrm{C})$. Axial current $I_\mathrm{R}$ flows
in the right hand compartment under the influence of the potential
difference $(V_\mathrm{C}-V_\mathrm{R})$. Transmembrane current
flow from the left and right hand circuits to the extracellular
medium at the endpoints of the left and right hand compartments.}}
\end{figure}

The accuracy of the two compartmental models can be assessed by
comparing their response to deterministic input for situations in
which the potential distribution in the model dendrite is known
analytically. The objective of this report is to examine the error
in the potential at the soma of a branched neuron in response to a
sustained input randomly placed on the dendritic tree. This
procedure will be repeated 2000 times and because of the linearity
of the model, the results of the simulation study will mimic the
effect of large scale exogenous input on the branched neuron. The
accuracy and precision of the models are assessed by comparing
their responses to this input in situations in which the potential
at the soma is known analytically.
