\section{Distributed and point process input to a segment}\label{dpi}

In general, segments receive distributed and point process sources
of input each of which require a different mathematical treatment.
The current supplied by distributed input such as intrinsic
voltage-dependent current or capacitative current is proportional
to the surface area of the segment on which it acts, whereas the
current supplied to a segment at a synapse or by an exogenous
point input is independent of the size of the segment. An implicit
assumption of a compartmental model is that distributed current
input to a segment is small by comparison with axial current
flowing along the segment.

To appreciate why this assumption is reasonable, consider a
cylindrical dendritic segment of radius $r$ (cm), length $h$ (cm)
and with membrane of constant conductance $g_\mathrm{M}$
(mS/cm$^2$). Suppose that axoplasm has constant conductance
$g_\mathrm{A}$ (mS/cm) and that a potential difference $V$ (mV)
exists between the segment boundaries, then the axial current
along the segment is $I_\mathrm{A}=\pi r^2 g_\mathrm{A} V/h$
($\mu$A) and the total distributed current crossing the membrane
of the segment is $I_\mathrm{M}=2\pi r h g_\mathrm{M}\,(V/2)$. The
ratio of the distributed current to the axial current is therefore
\begin{equation}\label{pc1}
\frac{\mbox{Distributed current}}{\mbox{Axial current}}
=\frac{I_\mathrm{M}}{I_\mathrm{A}}=\frac{\pi r h g_\mathrm{M}\,V}
{\pi r^2 g_\mathrm{A}\,(V/h)}=\frac{h^2 g_\mathrm{M}} {r
g_\mathrm{A}}=\Big(\frac{h}{r}\Big)^2\, \frac{r
g_\mathrm{M}}{g_\mathrm{A}}\,.
\end{equation}
For a typical dendritic segment $r g_\mathrm{M}/g_\mathrm{A}$ is
small (say $ \approx 10^{-5}$), and therefore distributed current
acting on a segment is small by comparison with axial current for
``short'' segments. On the other hand, segments several orders of
magnitude longer than their radius can be expected to have
distributed and axial currents of similar magnitude. An important
property of a compartmental model is that segments are not
excessively long by comparison with their radius. (However, see
Segev and Burke, \cite{Segev98}, Figure 3.3b). In the treatment of
distributed current, the development of the new compartmental
model makes explicit use of the assumption that distributed
current is much smaller than axial current. This assumption may
not be valid for point process input, and will not be made for the
treatment of this type of input in the new approach to
compartmental modelling.

\subsection{Axial current in the absence of transmembrane current}

The importance of the conclusion from Section \ref{dpi} is that
distributed transmembrane current acting on short segments is
small compared with axial current, and may be neglected in a first
approximation of the distribution of membrane potential on a
segment. Thus in the absence of point process input, the axial
current in a segment is well approximated from the potential drop
across the segment. In the light of this approximation, consider
Figure \ref{model} which illustrates a dendritic segment of length
$h$ in which $\lambda\in[0,1]$ is the fractional distance of a
point of the segment from its proximal end ($\lambda=0$). Let
$r_\mathrm{P}$ and $r_\mathrm{D}$ be the radii of the segment at
its proximal and distal boundaries respectively, let
$V_\mathrm{P}(t)$ and $V_\mathrm{D}(t)$ be the membrane potentials
at these boundaries and let $I_\mathrm{PD}$ be the axial current
in the segment in the absence of transmembrane current.

\begin{figure}[!h]
\centering
\begin{tabular}{c}
\includegraphics[ ]{NCFig2.eps}
\end{tabular}\qquad
\begin{tabular}{p{2.55in}}
\caption{\label{model} A segment of length $h$ is illustrated. In
the absence of transmembrane current, membrane potentials
$V_\mathrm{P}$ and $V_\mathrm{D}$ at the proximal and distal
boundaries of the segment generate axial current $I_\mathrm{PD}$.}
\end{tabular}
\end{figure}

%\begin{figure}[!h]
%\centering
%\begin{tabular}{c}
%\begin{mfpic}[1][1]{-40}{140}{180}{300}
%\headlen7pt
%\pen{0.5pt}
%\dotspace=4pt
%\dotsize=1pt
%\pen{1pt}
%\dotspace=4pt
%\dotsize=1.5pt
%%
%% LH cylinder
%\parafcn[s]{-180,180,5}{(100-21*sind(t),240+28*cosd(t))}
%\lines{(0,288),(100,268)}
%\lines{(0,192),(100,212)}
%%
%% Partial cylinder on left
%\dotted\parafcn[s]{0,180,5}{(36*sind(t),240+48*cosd(t))}
%\parafcn[s]{0,180,5}{(-36*sind(t),240+48*cosd(t))}
%%
%% Annotation of LH cylinder
%\dashed\arrow\lines{(0,240),(100,240)}
%\tlabel[bl](50,250){\large $I_\mathrm{PD}$}
%\tlabel[bc](0,250){$V_\mathrm{P}$}
%\tlabel[cc](0,180){\large $\lambda=0$}
%\tlabel[bc](0,295){\textsf{P}}
%\arrow\lines{(0,235),(0,200)}
%\tlabel[cr](-5,220){\textsf{$r_\mathrm{P}$}}
%%
%% Annotation of RH cylinder
%\tlabel[bc](100,250){$V_\mathrm{D}$}
%\tlabel[cc](100,180){\large $\lambda=1$}
%\tlabel[cc](100,280){\textsf{D}}
%\arrow\lines{(100,235),(100,216)}
%\tlabel[cl](105,228){\textsf{$r_\mathrm{D}$}}
%\arrow\lines{(60,228),(95,228)}
%\arrow\lines{(40,228),(5,228)}
%\tlabel[cc](50,228){$h$}
%\end{mfpic}
%\end{tabular}\qquad
%\begin{tabular}{p{2.55in}}
%\caption{\label{model} A segment of length $h$ (cm) is illustrated. In
%the absence of transmembrane current, membrane potentials $V_\mathrm{P}$
%and $V_\mathrm{D}$ at the proximal and distal boundaries of the
%segment generate axial current $I_\mathrm{PD}$.}
%\end{tabular}
%\end{figure}

The membrane of the segment in Figure \ref{model} is formed by
rotating the straight line PD about the axis of the dendrite to
form the frustum of a cone of radius
\begin{equation}\label{mp1}
r(\lambda)=(1-\lambda)r_\mathrm{P}+\lambda r_\mathrm{D} \,,\qquad
\lambda\in[0,1]\,.
\end{equation}
Assuming that the segment is filled with axoplasm of constant
conductance $g_\mathrm{A}$ and that no current crosses its
membrane, then the relationship between $V_\mathrm{P}$,
$V_\mathrm{D}$ and $I_\mathrm{PD}$ can be constructed by
integrating the defining equation of axial current, namely
$I_\mathrm{PD}=-g_\mathrm{A}\,A(x)\,dV/dx$, from the proximal to
the distal boundary of a segment. For the conical segment
illustrated in Figure \ref{model}, $A(x)=\pi r^2(\lambda)$,
$dV/dx=h^{-1}\,dV/d\lambda$ and the equation to be integrated is
\[
I_\mathrm{PD}=-\frac{g_\mathrm{A}\pi}{h}\,\Big[\,
(1-\lambda)r_\mathrm{P}+\lambda r_\mathrm{D}\,\Big]^2\,
\frac{dV}{d\lambda}
\]
with boundary conditions $V(0)=V_\mathrm{P}$ and
$V(1)=V_\mathrm{D}$. The result of this calculation is that the
the axial current $I_\mathrm{PD}$ and the potentials
$V_\mathrm{P}$ and $V_\mathrm{D}$ are connected by the formula
\begin{equation}\label{mp2}
I_\mathrm{PD}= \frac{\pi g_\mathrm{A} r_\mathrm{P}
r_\mathrm{D}}{h}\,\big(\,V_\mathrm{P}-V_\mathrm{D}\,\big)
\end{equation}
in the absence of transmembrane current. Moreover, the potential
at the point $\lambda$ is
\begin{equation}\label{mp3}
V(\lambda) = \frac{V_\mathrm{P}\,(1-\lambda)\,
r_\mathrm{P}+V_\mathrm{D}\,\lambda\,r_\mathrm{D}}
{(1-\lambda)\,r_\mathrm{P}+\lambda\,r_\mathrm{D}}\,.
\end{equation}
Note that equation (\ref{mp3}) is valid for sections with taper
and in the absence of taper will lead to a membrane potential
which varies linearly along the length of a segment. The
subsequent development of the new compartmental model assumes that
sections may taper unless stated specifically that the section is
uniform.

\subsection{Partitioning rule for transmembrane current}

In compartmental modelling the effect of transmembrane current is
represented in the model by input at points, or nodes, at which
the membrane potential is known. In a traditional approach to
compartmental modelling, these nodes are at the centre of
segments, whereas in the new approach they are located at the
boundaries of segments. The new approach partitions the effect of
input at any location between the nodes at the proximal and distal
boundaries of the segment. This procedure ensures that the
solution of the compartmental model is insensitive to small
changes in the location of segment boundaries because changes in
these boundaries also affect how the input is partitioned between
nodes. In the mathematical description of the new compartmental
model, the effect of input to a segment is treated as
perturbations $I_\mathrm{P}$ and $I_\mathrm{D}$ to the axial
current $I_\mathrm{PD}$ at the proximal and distal boundaries of a
segment. Axial current $I_\mathrm{PD}+I_\mathrm{P}$ is assumed to
leave the proximal boundary of a segment in the direction of its
distal boundary, while axial current $I_\mathrm{PD}+I_\mathrm{D}$
is assumed to arrive at the distal boundary of a segment from the
direction of its proximal boundary. The perturbations
$I_\mathrm{P}$ and $I_\mathrm{D}$ must satisfy the conservation of
current condition
\begin{equation}\label{potc1}
(I_\mathrm{PD}+I_\mathrm{D})-(I_\mathrm{PD}+I_\mathrm{P})+h\int_0^1
J(\lambda,t)\,d\lambda=0\quad\rightarrow\quad
I_\mathrm{P}-I_\mathrm{D}=h\int_0^1 J(\lambda,t)\,d\lambda
\end{equation}
where $J(\lambda,t)$ denotes transmembrane current. The task is to
construct expressions for $I_\mathrm{P}$ and $I_\mathrm{D}$ that
satisfy (\ref{potc1}) for all constitutive forms for the current
density $J(\lambda,t)$. The new approach to compartmental
modelling requires a procedure or rule for partitioning
transmembrane current between the proximal and distal boundaries
of a segment. The rule used in this article is that transmembrane
current flow to a boundary of a segment is proportional to the
axial conductance of the segment lying between the point of
application of the current and that boundary. If
$G_\mathrm{P}(\lambda)$ is the axial conductance of the portion of
segment lying between the point $\lambda$ and the proximal
boundary of the segment, and $G_\mathrm{D}(\lambda)$ is the axial
conductance of the portion of segment lying between the point
$\lambda$ and the distal boundary of the segment, then
\begin{equation}\label{potc2}
G_\mathrm{P}(\lambda) =
\frac{\pi g_\mathrm{A} r_\mathrm{P}r(\lambda)}{\lambda h}\,,\qquad
G_\mathrm{D}(\lambda) =
\frac{\pi g_\mathrm{A} r_\mathrm{D} r(\lambda)}{(1-\lambda)h}
\end{equation}
and the rule for partitioning transmembrane current leads to the
expressions
\begin{equation}\label{potc3}
\begin{array}{rcl}
I_\mathrm{P} & = & \ds h\int_0^1 \frac{G_\mathrm{P} J(\lambda,t)\,d\lambda}
{G_\mathrm{P}+G_\mathrm{D}}
= h\int_0^1 \frac{(1-\lambda)\,r_\mathrm{P}\,
J(\lambda,t)\,d\lambda}{(1-\lambda)\,r_\mathrm{P}
+\lambda\,r_\mathrm{D}}\,,\\[10pt]
-I_\mathrm{D} & = & \ds h\int_0^1 \frac{G_\mathrm{D} J(\lambda,t)\,d\lambda}
{G_\mathrm{P}+G_\mathrm{D}}
= h\int_0^1\frac{\lambda\,r_\mathrm{D}\,J(\lambda,t)\,d\lambda}
{(1-\lambda)\,r_\mathrm{P}+\lambda\,r_\mathrm{D}}\,.
\end{array}
\end{equation}
Clearly these expressions satisfy identically condition (\ref{potc1}) for the
conservation of current.

\subsection{Specification of transmembrane current}

Transmembrane current is usually assumed to consist of four
distinct components: capacitative current, intrinsic
voltage-dependent current and point process input which is
subdivided into synaptic current and exogenous point current.
Total transmembrane current is represented by
\begin{equation}\label{tc1}
\int 2\pi r \,c_\mathrm{M}\,\frac{\partial V}{\partial t}\,dx
+\int 2\pi r\,J_\mathrm{IVDC}(V)\,dx+\sum
J_\mathrm{SYN}(V_\mathrm{syn}) +\sum I_\mathrm{EX}
\end{equation}
where the integrals and summations are taken over the length of a
segment. In this expression $c_\mathrm{M}$ ($\mu$F/cm$^2$) is the
specific capacitance of the segment membrane, $V(x,t)$ is the
distribution of membrane potential at time $t$ (msec),
$J_\mathrm{IVDC}(V)$ ($\mu$A/cm$^2$) is the density of
transmembrane current due to intrinsic voltage-dependent channel
activity, $J_\mathrm{SYN}(V_\mathrm{syn})$ ($\mu$A) describes
synaptic input and $I_\mathrm{EX}$ ($\mu$A) describes exogenous
input. Although the specific capacitance of dendritic membrane is
normally taken to be constant in neuronal modelling, it will be
treated here as a function of position to show how transmembrane
current of this type may be incorporated into the new
compartmental model. For a segment of length $h$, the expression
for $J(\lambda,t)$ corresponding to formula (\ref{tc1}) is
\begin{equation}\label{tc2}
\begin{array}{rcl}
h J(\lambda,t) & = & \ds 2\pi h
r(\lambda)\,c_\mathrm{M}(\lambda)\,\frac{\partial
V(\lambda,t)}{\partial t}+2\pi h r(\lambda)\,J_\mathrm{IVDC}(V(\lambda,t))\\[10pt]
&&\qquad \ds+\;\sum_k
J_\mathrm{SYN}(V_\mathrm{syn})\,\delta(\lambda-\lambda_k) + \sum_k
I_\mathrm{EX}(t)\,\delta(\lambda-\lambda_k)
\end{array}
\end{equation}
where $\lambda_k$ denotes the relative location of the $k^{th}$
synapse or exogenous input with respect to the proximal boundary
of the segment ($\lambda=0$).
