\section{Construction of compartmental models}
The traditional and generalised compartmental models start with an
identical partitioning of a neuron into segments which interact
with the extracellular region (usually taken to be zero potential)
by means of transmembrane current driven by the potential of the
segment. In the mathematical model, each segment becomes a
compartment with electrical properties chosen to reflect the
biophysical properties of the segment it represents. Input to the
segment becomes input to the compartment, and shapes its potential
through a series of mathematical equations based on Kirchhoff's
circuit laws, and the fact that compartments only interact with
their nearest neighbours.

The traditional compartmental model treats each segment as an
iso-potential region of dendrite (\emph{e.g.} Segev and Burke,
\cite{Segev98}), and therefore all spatially distributed input
falling onto that segment experiences the same potential in the
model. Consequently, the electrical effect of spatial distributed
input on a segment is lost, and in practice all input to a segment
is assumed to act at a single point, conventionally taken to be
the centre of the segment.

In the generalised compartmental model of a neuron the membrane
potential of a segment is allowed to vary along its length
allowing  the solution for the membrane potential in the
generalised model to more accurately reflect the influence of
spatially distributed input. While this might appear to be a minor
difference between both models, and of no great consequence, it
will be shown by simulation that this modification improves both
the accuracy and precision of the compartmental model by an order
of magnitude. Furthermore, this improvement will be achieved
without any significant increase in computational effort. To
appreciate the consequences of a non-constant compartmental
potential, the steps in the construction of the traditional
compartmental model are reviewed briefly before a detailed
description of the generalised model is presented.

\subsection{Geometrical construction of dendritic segments}
Compartmental models of the neuron usually regard the dendritic
section as a sequence of contiguous tapered circular
cylinders\footnote{ One widely used implementation of
compartmental models is the NEURON simulator developed by Hines
and Carnevale (\cite{Hines97}).}. Any point at which a noticeable
change in taper occurs is a natural candidate for the designated
node of a dendritic segment, by this we mean the node at which the
membrane potential of that segment is recorded. In practice, the
designated nodes of a segment will correspond to the points at
which morphological information is available. Figure \ref{2d}
shows a dendritic segment
$\mathcal{C}=[x_\mathrm{P},x_\mathrm{Q}]$ where $x_\mathrm{P}$ and
$x_\mathrm{Q}$ are the respective distances of the left hand and
right hand endpoints of the segment from the somal end of the
dendritic section.

\begin{figure}[!h]
\[
\begin{array}{c}
$\begin{mfpic}[1.1][1.1]{-30}{150}{155}{320}
\pen{1pt}
\headlen4pt
\arrow\lines{(15,300),(5,300)}
\arrow\lines{(80,300),(90,300)}
\headlen7pt
\dotspace=4pt
\dotsize=1.5pt
%
% Central cylinder
\lines{(100,272),(130,272)}
\lines{(100,208),(130,208)}
%
% LH cylinder
\dotted\parafcn[s]{0,180,5}{(100+24*sind(t),240+32*cosd(t))}
\parafcn[s]{0,180,5}{(100-24*sind(t),240+32*cosd(t))}
\lines{(0,288),(100,272)}
\lines{(0,192),(100,208)}
%
% Partial cylinder on left
\dotted\parafcn[s]{0,180,5}{(36*sind(t),240+48*cosd(t))}
\parafcn[s]{0,180,5}{(-36*sind(t),240+48*cosd(t))}
\lines{(-30,288),(0,288)}
\lines{(-30,192),(0,192)}
%
% Annotation of LH cylinder
\arrow\lines{(30,240),(60,240)}
\tlabel[bl](55,250){$I_\mathrm{PQ}$}
\tlabel[bc](0,250){$V_\mathrm{P}$}
\tlabel[cc](0,300){\large $x_\mathrm{P}$}
\tlabel[cc](0,240){\large $\bullet$}
\arrow\lines{(0,230),(0,170)}
\tlabel[tc](0,165){$I^{(m)}_\mathrm{P}$}
%
% Annotation of Central cylinder
\dotted\parafcn[s]{0,180,5}{(50+30*sind(t),240+40*cosd(t))}
\parafcn[s]{0,180,5}{(50-30*sind(t),240+40*cosd(t))}
\tlabel[bc](100,250){$V_\mathrm{Q}$}
\tlabel[cc](100,300){\large $x_\mathrm{Q}$}
\tlabel[cc](100,240){\large $\bullet$}
\arrow\lines{(100,230),(100,170)}
\tlabel[tc](100,165){$I^{(m)}_\mathrm{Q}$}
%
% Form compartment
\tlabel[cc](50,300){\textsf{Compartment}}
\end{mfpic}$
\end{array}\qquad
\begin{tabular}{p{1.9in}}
\caption{\label{2d}
The compartment occupying $[x_\mathrm{P},x_\mathrm{Q}]$ is
represented. Currents $I^{(m)}_\mathrm{P}$ and
$I^{(m)}_\mathrm{Q}$ flow across the membrane at $x_\mathrm{P}$
and $x_\mathrm{Q}$ respectively and axial current $I_\mathrm{PQ}$
flows from $x_\mathrm{P}$ to $x_\mathrm{Q}$ through the resistive
dendritic core.}
\end{tabular}
\]
\end{figure}

Let $r_\mathrm{P}$ and $r_\mathrm{Q}$ be the respective radii of
the dendrite at the nodes $x_\mathrm{P}$, and $x_\mathrm{Q}$. When
the dendritic membrane is formed by rotating the straight line
segment $[x_\mathrm{P},x_\mathrm{Q}]$ about the axis of the
dendrite, as illustrated in Figure \ref{2d}, the membrane of the
segment has radius
\begin{equation}\label{car2}
r(x) = \frac{r_\mathrm{P}(x_\mathrm{Q}-x)
+r_\mathrm{Q}(x-x_\mathrm{P})}{x_\mathrm{Q}-x_\mathrm{P}}\,,
\qquad x\in\mathcal{C}\,.
\end{equation}
Moreover, it is straight forward Calculus to verify that
\begin{equation}\label{dc1}
\int_\mathcal{C}\,r(x)\,dx=
\frac{1}{2}\,\big(x_\mathrm{Q}-x_\mathrm{P}\big)
\big(r_\mathrm{P}+r_\mathrm{Q}\big)\,.
\end{equation}

\subsection{Intracellular resistance of dendritic segments}
Assuming that the intracellular medium of the dendrite has
constant conductance $g_\mathrm{A}$ (mS/cm), the general
expressions for the axial resistance of the segment illustrated in
Figure \ref{2d} is
\begin{equation}\label{car1}
R_\mathrm{PQ} = \frac{1}{g_\mathrm{A}}\,
\int_{\mathcal{C}}\,\frac{dx}{\pi r^2(x)}=
\frac{x_\mathrm{Q}-x_\mathrm{P}}
{\pi g_\mathrm{A} r_\mathrm{P}r_\mathrm{Q}}
\end{equation}
where $r(x)$ is the radius of the dendritic section at position
$x$, and the value stated in (\ref{car1}) is that for the
piecewise tapered dendritic segment with radius given by
(\ref{car2}). If $V_\mathrm{L}$ and $V_\mathrm{Q}$ are the
respective potentials at the endpoints $x_\mathrm{P}$ and
$x_\mathrm{Q}$ of the segment $\mathcal{C}$ then, in the absence
of transmembrane current across the segment, the currents
$I_\mathrm{PQ}$ appearing in Figure \ref{2d} is determined by
Ohm's law and has value
\begin{equation}\label{car4}
I_\mathrm{PQ} = \ds\frac{(V_\mathrm{P}-V_\mathrm{Q})}
{R_\mathrm{PQ}} = \frac{\pi g_\mathrm{A}r_\mathrm{P}r_\mathrm{Q}}
{x_\mathrm{Q}-x_\mathrm{P}}\;\big(V_\mathrm{P}-V_\mathrm{Q}\big)\,.
\end{equation}
The dependence of $V_\mathrm{P}$ and $V_\mathrm{Q}$ on time $t$
has been suppressed in equation (\ref{car4}) for representational
simplicity, but is will be understood henceforth that all
potentials at segment endpoints are functions of time although the
dependence on $t$ is not made explicit. Most importantly, the
calculation which leads to expressions (\ref{car1}) and
(\ref{car4}) for the compartment resistance and axial current also
implies that the potential distribution within the segment is
\begin{equation}\label{tc1}
V(x,t) = V_\mathrm{P}-I_\mathrm{PQ}\,\int^x_{x_\mathrm{P}}
\;\frac{ds}{\pi g_\mathrm{A}\,r^2(s)}\,,\qquad
x\in\mathcal{C}\,.
\end{equation}
For a dendritic section that is uniformly tapered this
distribution is
\begin{equation}\label{tc2}
V(x,t) = \frac{V_\mathrm{P}\,r_\mathrm{P}\,(x_\mathrm{Q}-x)+
V_\mathrm{Q}\,r_\mathrm{Q}(x-x_\mathrm{P})}{r(x)\,
(x_\mathrm{Q}-x_\mathrm{P})}\,,\qquad x\in\mathcal{C}\,.
\end{equation}
Furthermore, the average potential of the compartment as measured
in terms of the charge carrying capacity of its membrane is
defined by
\begin{equation}\label{tc3}
V_\mathcal{C}=\frac{\ds\int_\mathcal{C}\,r(x)V(x,t)\,dx}
{\ds\int_\mathcal{C}\,r(x)\,dx}=
\frac{V_\mathrm{P}\,r_\mathrm{P}+V_\mathrm{Q}\,r_\mathrm{Q}}
{r_\mathrm{P}+r_\mathrm{Q}}\,.
\end{equation}

Both the traditional and generalised compartmental models are
based on Kirchhoff's current law which asserts that $I_\mathrm{LC}
= I^{(m)}_\mathrm{C}+I_\mathrm{CR}$. Thus
\begin{equation}\label{car5}
\frac{V_\mathrm{L}}{R_\mathrm{LC}}
-\Big(\,\frac{V_\mathrm{C}}{R_\mathrm{LC}}
+\frac{V_\mathrm{C}}{R_\mathrm{CR}}\,\Big)
+\frac{V_\mathrm{R}}{R_\mathrm{CR}}
=I^{(m)}_\mathrm{C}\,,
\end{equation}
or in terms of the dendritic radii,
\begin{equation}\label{car6}
\Big(\frac{\pi g_\mathrm{A}r_\mathrm{L}r_\mathrm{C}}
{x_\mathrm{C}-x_\mathrm{L}}\Big)\;V_\mathrm{L}
-\Big(\frac{\pi g_\mathrm{A}r_\mathrm{L}r_\mathrm{C}}
{x_\mathrm{C}-x_\mathrm{L}}
+\frac{\pi g_\mathrm{A}r_\mathrm{C}r_\mathrm{R}}
{x_\mathrm{R}-x_\mathrm{C}}\Big)\;V_\mathrm{C}
+\Big(\frac{\pi g_\mathrm{A}r_\mathrm{C}r_\mathrm{R}}
{x_\mathrm{R}-x_\mathrm{C}}\Big)\;V_\mathrm{R}=I^{(m)}_\mathrm{C}\,.
\end{equation}

\subsection{Specification of transmembrane current}
The current crossing the membrane of the dendritic segment in
Figure \ref{2d} has general expression
\begin{equation}\label{tc0}
\begin{array}{rcl}
I^{(m)} & = & \ds 2\pi c_\mathrm{M}\,\frac{d}{dt}\,
\int_\mathcal{C}\, r(x)\,V(x,t)\,dx
+2\pi\int_\mathcal{C}\, r(x)J_\mathrm{IVDC}(V(x,t))\,dx\\[12pt]
&&\quad\ds+\;\sum_\mathcal{C}\,J_\mathrm{SYN}(V(x,t))
+\sum_\mathcal{C}\,I_\mathrm{EX}(x,t)
\end{array}
\end{equation}
where $c_\mathrm{M}$ ($\mu$F/cm$^2$) is the specific capacitance
(assumed constant) of the segment membrane, $V(x,t)$ is the
membrane potential of the dendritic section at time $t$ (ms) and
distance $x$ from its somal end, $J_\mathrm{IVDC}(x,t)$ is the
density of transmembrane current ($\mu$A/cm$^2$) due to intrinsic
voltage-dependent channel activity, $J_\mathrm{SYN}(x,t)$ is the
linear density of synaptic current ($\mu$A/cm) due to synaptic
activity falling on the segment and $I_\mathrm{EX}(x,t)$ is the
linear density of exogenous current ($\mu$A/cm). The difference
between the traditional and generalised compartmental models lies
in the mathematical representation of $I^{(m)}$.

\section{The traditional compartmental model}
In the traditional compartmental model, the compartment is assumed
to be iso-potential with membrane potential $V_\mathcal{C}$, the
average potential of the compartment. With this assumption, the
transmembrane current in the traditional compartmental model
simplifies to
\begin{equation}\label{tc00}
I^{(m)} = \pi\big(x_\mathrm{Q}-x_\mathrm{P}\big)
\big(r_\mathrm{P}+r_\mathrm{Q}\big)\,\Big[\,c_\mathrm{M}\,
\frac{dV_\mathcal{C}}{dt}\,+J_\mathrm{IVDC}(V_\mathrm{C})\,\Big]
+\sum_\mathcal{C}\,J_\mathrm{SYN}(V_\mathrm{C})
+\sum_\mathcal{C}\,I_\mathrm{EX}(x,t)\,.
\end{equation}
For segments constructed from piecewise tapered elements,
the values of these integrals are given in equations (\ref{dc1}).

\subsection{The model differential equations}
To make explicit the essential features of the mathematical
problem that must be solved when using the traditional
compartmental model to describe neuronal behaviour, it is
necessary to state how intrinsic voltage-dependent current and
synaptic current are to be modelled. The most common description
of intrinsic voltage-dependent current is due to Hodgkin and
Huxley (\cite{Hodgkin52}) and assumes that the contribution to
transmembrane current density from ionic channels of species
$\alpha$ is $J_\mathrm{IVDC}=g_\alpha(V)(V-E_\alpha)$ where
$E_\alpha$ is the reversal potential for the species $\alpha$. The
conductance $g_\alpha$ is defined in terms of auxiliary variables
which themselves satisfy differential equations with coefficients
that are dependent on the local membrane potential. It is in this
sense that the conductance $g_\alpha(V)$ is dependent on the
membrane potential. On the other hand, synaptic input due to ionic
species $\beta$ is modelled by the specification
$J_\mathrm{SYN}=g_\beta(t)(V-E_\beta)$ where $E_\beta$ is the
reversal potential for the species and $g_\beta(t)$ is now the
time course of the synaptic conductance. With these model
representations of intrinsic voltage-dependent current and
synaptic current, the transmembrane current at $x_\mathrm{C}$ has
generic form
\begin{equation}\label{mde1}
\begin{array}{rcl}
I^{(m)}_\mathrm{C} & = & \ds C_\mathrm{C}\,
\frac{dV_\mathrm{C}}{dt}\,+\sum_\alpha\;
G^\mathrm{IVDC}_\alpha(V_\mathrm{C})\big(\,V_\mathrm{C}-E_\alpha\,\big)
+\sum_\beta\,G^\mathrm{SYN}_\beta(t)
\big(\,V_\mathrm{C}-E_\beta\,\big)+I_\mathrm{C}(t)
\end{array}
\end{equation}
where $C_\mathrm{C}$ (constant) is the total membrane capacitance
of the segment, $G^\mathrm{IVDC}_\alpha(V_\mathrm{C})$ denotes the
total intrinsic voltage-dependent conductance of the channels of
ionic species $\alpha$ associated with the segment,
$G^\mathrm{SYN}_\beta(t)$ is the total synaptic conductance at
time $t$ associated with channels of ionic species $\beta$ falling
on the segment and $I_\mathrm{C}(t)$ plays the role of the total
exogenous current input to the segment at time $t$.

Suppose that the neuron is partitioned into $m$ compartments where
the membrane potential at the designated node of the $k^{th}$
compartment is $V_k(t)$ and let
\begin{equation}\label{mde2}
V(t)=\big[\,V_1(t),V_1(t),\cdots,V_m(t)\,]^\mathrm{T}\,.
\end{equation}
It follows immediately from the expression (\ref{mde1}) for the
transmembrane current $I^{(m)}_\mathrm{C}$ and (\ref{car5}) that
$V(t)$, the column vector of membrane potentials, satisfies the
ordinary differential equations
\begin{equation}\label{mde3}
D^\mathrm{C}\,\frac{dV}{dt}+D^\mathrm{IVDC}(V)\,V+D^\mathrm{SYN}(t)\,V
+I(t)=AV
\end{equation}
where $D^\mathrm{C}$ is a constant diagonal matrix,
$D^\mathrm{IVDC}(V)$ is a diagonal matrix of intrinsic
voltage-dependent conductances, $D^\mathrm{SYN}(t)$ is a diagonal
matrix of synaptic conductances and $I(t)$ is a column vector of
exogenous currents. The $(j,k)^{th}$ entry of the matrix $A$,
which is interpreted as a conductance matrix, is nonzero if the
$j^{th}$ and $k^{th}$ designated nodes are neighbours, otherwise
the entry is zero. The computational complexity of the final
mathematical problem is determined by the structure of $A$
provided all other matrices arising in the mathematical
specification of the problem are not more complex than $A$. This
is certainly true for the traditional compartmental model since
all matrices other than $A$ are diagonal. Integration of equation
(\ref{mde3}) over the interval $[t,t+h]$ yields
\begin{equation}\label{mde4}
D^\mathrm{C}\,\big[\,V(t+h)-V(t)\,\big]+
\int_t^{t+h}\,\big[\,D^\mathrm{IVDC}(V)+
D^\mathrm{SYN}(t)\,\big]\,V(t)\,dt
+\int_t^{t+h}\,I(t)=A\int_t^{t+h}\,V(t)\,dt\,.
\end{equation}
The trapezoidal quadrature is used to replace each integral in
equation (\ref{mde4}) with the exception of the integral of
intrinsic voltage-dependent current, which is replaced by the
midpoint quadrature. The result of this calculation is
\begin{equation}\label{mde5}
\begin{array}{l}
\ds D^\mathrm{C}\,\big[\,V(t+h)-V(t)\,\big]+
h\,D^\mathrm{IVDC}(V(t+h/2))\,V(t+h/2)\\[10pt]
\quad\ds+\;\frac{h}{2}\,\Big[\,D^\mathrm{SYN}(t+h)V(t+h)
+D^\mathrm{SYN}(t)V(t)\,\Big]+\frac{h}{2}
\Big[\,I(t+h)+I(t)\,\Big]\\[10pt]
\qquad\ds = \frac{h}{2}\,\Big[\,A V(t+h)+AV(t)\,\Big]+O(h^3)\,.
\end{array}
\end{equation}
On taking account of the fact that
\[
V(t+h/2)=\frac{1}{2} \,\Big[\,V(t+h)+V(t)\,\Big]+O(h^2)\,,
\]
equation (\ref{mde5}) may be reorganised to give
\begin{equation}\label{mde6}
\begin{array}{l}
\ds \Big[\,2 D^\mathrm{C}-hA+h\,D^\mathrm{SYN}(t+h)
+h\,D^\mathrm{IVDC}(V(t+h/2))\,\Big]\,V(t+h) = \\[10pt]
\qquad\ds \Big[\,2 D^\mathrm{C}+hA+h D^\mathrm{SYN}(t)
-h\,D^\mathrm{IVDC}(V(t+h/2))\,\Big]\,V(t)
-h\Big[\,I(t+h)+I(t)\,\Big]
\end{array}
\end{equation}
when the error structure is ignored. The detailed computation of
$D^\mathrm{IVDC}(V(t+h/2))$ is determined entirely by the
structure of the auxiliary equations. In the case of
Hodgkin-Huxley like channels, it is standard knowledge that
$D^\mathrm{IVDC}(V(t+h/2))$ can be computed to adequate accuracy
from $V(t)$ and the differential satisfied by the auxiliary
variables (Lindsay \emph{et al.}, \cite{Lindsay01a}).

\section{The generalised compartmental model}
Equations (\ref{tc1}) and (\ref{tc2}) provide the basis for the
construction of the transmembrane current in the generalised
compartmental model. Within the framework of this model,
distributed transmembrane current and point sources of
transmembrane current receive a different mathematical treatment.

To appreciate why this is the case, consider a cylindrical
dendritic segment of radius $a$, length $L$ and with membrane of
constant conductance $g_\mathrm{M}$. Suppose that the segment is
filled with intracellular medium of conductance $g_\mathrm{A}$ and
that a potential difference $V$ exists across its length $L$. The
axial current flowing along the segment is $I_\mathrm{A}=\pi a^2
g_\mathrm{A} V/L$ and the total transmembrane current is
$I_\mathrm{M}=2\pi a L g_\mathrm{M}\,(V/2)$. Thus
\begin{equation}\label{pc1}
\frac{\mbox{Transmembrane current}}{\mbox{Axial current}}
=\frac{I_\mathrm{M}}{I_\mathrm{A}}=\frac{\pi a L g_\mathrm{M}\,V}
{\pi a^2 g_\mathrm{A}\,(V/L)}=\frac{L^2 g_\mathrm{M}}
{a g_\mathrm{A}}=\Big(\frac{L}{a}\Big)^2\,
\frac{a g_\mathrm{M}}{g_\mathrm{A}}\,.
\end{equation}
For a typical dendrite $a g_\mathrm{M}/g_\mathrm{A}\approx
10^{-5}$ which in turn suggest that membrane current losses are
comparable to axial current only provided the segment is several
orders of magnitude longer than its radius. Since the model only
allows current to flow across the membrane at designated nodes, a
well-structured compartmental model requires that the internodal
distance does not become several orders of magnitude greater than
the radius of the dendrite. By meeting this requirement, the
effect of the transmembrane current on the axial current is
locally negligible.

Provided a compartment is not excessively long, the implication of
this argument is that distributed transmembrane current has
negligible impact on the local axial current flowing between
designated nodes. Consequently, the effect of this distributed
transmembrane current may be described in terms of the membrane
potential computed from the axial current by ignoring the
transmembrane current itself. On the other hand, point input of
current due to synaptic activity or exogenous input necessarily
causes a discontinuity in axial current irrespective of the size
of the compartment or the strength of the input. Consequently,
point current input must be treated separately from distributed
input because it necessarily generates discontinuities in axial
current and therefore affects the local transmembrane potential.

The treatment of distributed transmembrane current takes advantage
of the identity
\begin{equation}\label{gcm1}
\begin{array}{rcl}
\ds\int_{x_\mathrm{P}}^{(x_\mathrm{P}+x_\mathrm{Q})/2}\,r(x)\,V(x,t)\,dx & = &
\ds\frac{1}{8}\big(x_\mathrm{Q}-x_\mathrm{P}\big)
\big(\,3r_\mathrm{P} V_\mathrm{P}+r_\mathrm{Q} V_\mathrm{Q}\,\big)\,,\\[12pt]
\ds\int_{(x_\mathrm{P}+x_\mathrm{Q})/2}^{x_\mathrm{Q}}\,r(x)\,V(x,t)\,dx & = &
\ds\frac{1}{8}\big(x_\mathrm{Q}-x_\mathrm{P}\big)
\big(r_\mathrm{P} V_\mathrm{P}+3r_\mathrm{Q}
V_\mathrm{Q}\,\big)\,.
\end{array}
\end{equation}
These equations describe how transmembrane current falling on the
segment should be partitioned between the membrane currents
$I^{(m)}_\mathrm{P}$ and $I^{(m)}_\mathrm{Q}$ crossing the
membrane at $x_\mathrm{P}$ and $x_\mathrm{Q}$.

\subsection{Capacitative current}
It now follows
immediately from the general expression (\ref{tc0}) for
transmembrane current that the capacitative component of this
current is
\begin{equation}\label{gcm2}
\begin{array}{l}
\ds 2\pi c_\mathrm{M}\,\frac{d}{dt}\,
\int_\mathcal{L}\, r(x)\,V(x,t)\,dx+2\pi c_\mathrm{M}\,
\frac{d}{dt}\,\int_\mathcal{R}\, r(x)\,V(x,t)\,dx = \\[12pt]
\qquad\ds\frac{\pi c_\mathrm{M}\big(x_\mathrm{C}-x_\mathrm{L}\big)}{2}
\Big[\,r_\mathrm{L}\frac{dV_\mathrm{L}}{dt}+3r_\mathrm{C}
\frac{dV_\mathrm{C}}{dt}\,\Big]+
\frac{\pi c_\mathrm{M}\big(x_\mathrm{R}-x_\mathrm{C}\big)}{2}
\Big[\,3r_\mathrm{C}\frac{dV_\mathrm{C}}{dt}+r_\mathrm{R}
\frac{dV_\mathrm{R}}{dt}\,\Big]\,.
\end{array}
\end{equation}

\subsection{Intrinsic voltage-dependent current}
A common specification of intrinsic voltage-dependent current
describing the behaviour of channels of ionic species $\alpha$
assumes that $J_\mathrm{IVDC}(x,t)=g_\alpha(V)(V-E_\alpha)$ where
$V$ is the transmembrane potential, $E_\alpha$ (mV) is the
reversal potential for species $\alpha$ and $g_\alpha(V)$
(mS/cm$^2$) is a voltage-dependent membrane conductance (which may
depend on a set of auxiliary variables such as the $m$, $n$ and
$h$ appearing in the Hodgkin-Huxley (\cite{Hodgkin52}) model). The
simplest case is the \emph{passive} membrane in which
$g_\alpha(V)$ is constant for each species $\alpha$, albeit a
different constant for each species. It then follows immediately
from identity (\ref{gcm1}) and the general expression (\ref{tc0})
that the contribution of intrinsic voltage-dependent current to
the segment due to species $\alpha$ in a passive membrane is
\begin{equation}\label{gcm3}
\begin{array}{l}
\ds 2\pi\int_\mathcal{L}\, r(x) g_\alpha(V-E_\alpha)\,dx
+2\pi\int_\mathcal{R}\, r(x) g_\alpha(V-E_\alpha)\,dx = \\[12pt]
\qquad\ds\frac{\pi\big(x_\mathrm{C}-x_\mathrm{L}\big)}{2}
\Big[\,r_\mathrm{L} g_\alpha(V_\mathrm{L}-E_\alpha)
+3r_\mathrm{C} g_\alpha(V_\mathrm{C}-E_\alpha)\,\Big]\\[12pt]
\qquad\qquad\ds+\;\frac{\pi\big(x_\mathrm{R}-x_\mathrm{C}\big)}{2}
\Big[\,3r_\mathrm{C} g_\alpha(V_\mathrm{C}-E_\alpha)
+r_\mathrm{R} g_\alpha(V_\mathrm{R}-E_\alpha)\,\Big]\,.
\end{array}
\end{equation}
On the other hand, when $g_\alpha(V)$ is a non-constant function
of $V$ as happens, for example, with a Hodgkin-Huxley membrane
(\cite{Hodgkin52}), no analytical expression for the effect of
intrinsic voltage transmembrane current exist. To resolve this
impasse, one requires a generalised expression to describe the
effect of intrinsic voltage-dependent transmembrane current. This
expression must have the following properties. First, it must be
tractable when $g_\alpha(V)$ is a non-constant function of $V$ in
the sense that $g_\alpha(V)$ is evaluated for values of $V$ at
designate nodes only. Second, the expression must incorporate the
effect of changing membrane potential along a dendritic segment,
and third, the generalised expression must reduce to expression
(\ref{gcm3}) when $g_\alpha(V)$ is a constant function of $V$. For
the specification $J_\mathrm{IVDC}(x,t)=g_\alpha(V)(V-E_\alpha)$,
these three conditions are satisfied by replacing
\[
\ds 2\pi\int_\mathcal{L}\, r(x) g_\alpha(V)(V-E_\alpha)\,dx
+2\pi\int_\mathcal{R}\, r(x) g_\alpha(V)(V-E_\alpha)\,dx
\]
with the generalised expression
\begin{equation}\label{gcm4}
\begin{array}{l}
\ds\frac{\pi\big(x_\mathrm{C}-x_\mathrm{L}\big)}{2}
\Big[\,r_\mathrm{L} g_\alpha(V_\mathrm{L})(V_\mathrm{L}-E_\alpha)
+3r_\mathrm{C} g_\alpha(V_\mathrm{C})(V_\mathrm{C}-E_\alpha)\,\Big]\\[12pt]
\qquad\ds+\;\frac{\pi\big(x_\mathrm{R}-x_\mathrm{C}\big)}{2}
\Big[\,3r_\mathrm{C} g_\alpha(V_\mathrm{C})(V_\mathrm{C}-E_\alpha)
+r_\mathrm{R} g_\alpha(V_\mathrm{R})(V_\mathrm{R}-E_\alpha)\,\Big]\,.
\end{array}
\end{equation}
Moreover, it is clear that the specification of intrinsic
voltage-dependent current used in the traditional compartmental
model is recovered from (\ref{gcm4}) by replacing $V_\mathrm{L}$
and $V_\mathrm{R}$ with $V_\mathrm{C}$.
