\documentclass[11pt]{article}
\usepackage{amsfonts,amsbsy, amssymb, graphicx}

% Tree-saver
\setlength{\textwidth}{8.276in}
\setlength{\textheight}{11.705in}

%Allow 1in margin on each side and nothing else
\addtolength{\textwidth}{-2in}
\addtolength{\textheight}{-2in}
\setlength{\oddsidemargin}{0pt}
\setlength{\evensidemargin}{\oddsidemargin}
\setlength{\topmargin}{0pt}
\addtolength{\topmargin}{-\headheight}
\addtolength{\topmargin}{-\headsep}

\input mfpic.tex

\renewcommand{\baselinestretch}{1.3}
\renewcommand{\mathrm}[1]{{\mbox{\tiny #1}}}

\newcommand{\dfrac}[2]{\displaystyle{\frac{#1}{#2}}}
\newcommand{\ds}[1]{\displaystyle#1}
\newcommand{\bs}[0]{\boldsymbol}

\parindent=0pt
\parskip=5pt

\title{\LARGE\bf Increased computational accuracy in
multi-compartmental cable models by a novel approach for precise
point process localization}
\author{\Large\bf A.E. Lindsay\\
Department of Mathematics, University of Edinburgh,\\
Edinburgh EH9 3JZ \\[10pt]
\Large\bf K.A. Lindsay \\
Department of Mathematics, University of Glasgow,\\
Glasgow G12 8QQ \\[10pt]
\Large\bf J.R. Rosenberg$^\dagger$\\
Division of Neuroscience and Biomedical Systems,\\
University of Glasgow, Glasgow G12 8QQ}

\makeatletter
\def\@cite#1#2{{#1\if@tempswa , #2\fi}}
\def\@biblabel#1{ }
%\def\@biblabel#1{#1.}
\makeatother

\begin{document}

\opengraphsfile{mfpic}

\maketitle

\thispagestyle{empty}

\vfil

\begin{tabular}{ll}
$^\dagger$  & \textbf{Corresponding author} \\[5pt]
            & J.R. Rosenberg \\
            & West Medical Building \\
            & Division of Neuroscience and Biomedical Systems \\
            & University of Glasgow \\
            & Glasgow G12 8QQ \\
            & Scotland UK \\[5pt]
            & Tel\quad(+44) 141 330 6589 \\
            & Fax\quad(+44) 141 330 2923 \\
            & Email \verb$j.rosenberg@bio.gla.ac.uk$\\[10pt]
            & \textbf{Keywords} \\[5pt]
            & Compartmental models, Dendrites, Cable Equation
\end{tabular}

\vfil

\pagebreak[4]

\begin{center}
\begin{tabular}{p{5.2in}}
\multicolumn{1}{c}{\textbf{Abstract}}\\[10pt]

Compartmental models of dendrites are the most widely used tool
for investigating their electrical behaviour. Traditional models
assign a single potential to a compartment. This potential is
associated with the membrane potential at the centre of the
segment represented by the compartment. All input to that segment,
independent of its location on the segment, is assumed to act at
the centre of the segment with the potential of the compartment.
By contrast, the compartmental model introduced in this article
assigns a potential to each end of a segment, and takes into
account the location of input to a segment on the model solution
by partitioning the effect of this input between the axial
currents at the proximal and distal boundaries of segments. For a
given neuron, the new and traditional approaches to compartmental
modelling use the same number of locations at which the membrane
potential is to be determined, and lead to ordinary differential
equations that are structurally identical. However, the solution
achieved by the new approach gives an order of magnitude better
accuracy and precision than that achieved by the latter in the
presence of point process input.
\end{tabular}
\end{center}

\pagebreak[4]

\input NC1.tex
\input NC2.tex
\input NC3.tex
\input NC4.tex
\input NC5.tex
\input NC6.tex

\closegraphsfile

\end{document}

Compartmental models have become important tools for investigating
the behaviour of neurons to the extent that a number of packages
exist to facilitate their implementation (\emph{e.g.} Hines and
Carnevale \cite{Hines97}; Bower and Beeman \cite{Bower97}). Their
use is motivated by the desire to reduce the mathematical
complexity inherent in a continuum description of a neuron. This
simplification is achieved by replacing the partial differential
equations defining the continuum description of a neuron by a
compartmental model of the neuron in which its behaviour is
described by the solution of a set of ordinary differential
equations (Rall, \cite{Rall64}).

The traditional approach to compartmental modelling, introduced by
Rall (\cite{Rall64}), assumes that a ``lump of membrane becomes a
compartment; the rate constants governing exchange between
compartments are proportional to the series conductance between
them". Rall's definition of a compartmental model thus
distinguishes between the input acting on a localised region of
neuronal membrane (the compartment) and the resistive properties
of the axoplasm which determines the conductances linking
compartments in his model. Other authors (\emph{e.g.} Segev and
Burke, \cite{Segev98}) treat the neuronal segment, including the
membrane and axoplasm, as the compartment. Both definitions,
however, associate a single potential with a compartment, and
assume that all input falling on the segment that is represented
by the compartment will act with this potential. For this reason
these compartments are iso-potential, and indeed Segev and Burke
(\cite{Segev98}) state this explicitly. Of course, iso-potential
compartments are a feature of the model and \emph{should not be
confused} with the true potential distribution within segments.

Compartmental models in which a compartment has a single potential
are aesthetically unsatisfactory since a compartment of this type
cannot act as the fundamental unit in the construction of a model
dendrite for two reasons. First, compartments defined by a single
potential must coexist in pairs in order to support axial current
flow, and second, half compartments are required to represent
branch points and dendritic terminals (\emph{e.g.}, Segev and
Burke, \cite{Segev98}). In the new approach to compartmental
modelling presented in this article, two potentials are assigned
to each compartment --- one to represent the membrane potential at
the proximal boundary of the segment and the other to represent
the membrane potential at its distal boundary. The new compartment
can exist as an independent entity without the need to introduce
half compartments, and can therefore function as the basic
building block of a multi-compartmental neuronal model. The new
compartments more accurately describe the influence of point
current and synaptic input to the segments they represent than
those of a traditional compartmental model.

The accuracy of the new and traditional approaches to
compartmental modelling is first assessed by calculating the error
in the somal potential of a test neuron when each approach is used
to calculate this potential ten milliseconds after the initiation
of large scale point current input. In a second comparison, the
accuracy of the two approaches is assessed by comparing the
statistics of the spike train output generated by each type of
compartmental model of the test neuron when subjected to large
scale synaptic input.
