\documentclass{slides}
\usepackage{amsfonts,amsbsy,amssymb,graphicx}

% Tree-saver
%\setlength{\textwidth}{11.705in}
%\setlength{\textheight}{8.276in}
\setlength{\textwidth}{8.276in}
\setlength{\textheight}{11.705in}
%Allow 1in margin on each side and nothing else
\addtolength{\textwidth}{-2in}
\addtolength{\textheight}{-2in}
\setlength{\oddsidemargin}{0pt}
\setlength{\evensidemargin}{\oddsidemargin}
\setlength{\topmargin}{0pt}
\addtolength{\topmargin}{-\headheight}
\addtolength{\topmargin}{-\headsep}

\input mfpic.tex

\newcommand{\dfrac}[2]{\displaystyle{\frac{#1}{#2}}}
\newcommand{\bs}[1]{\boldsymbol{#1}}
\def\ds{\displaystyle}
\def\ss{\,\scriptsize}


\begin{document}

\opengraphsfile{mfpic}

%
%
\begin{slide}
\begin{center}
{\bf Some new ideas in compartmental modelling}\\[60pt]
K.A. Lindsay, \\
Department of Mathematics, University of Glasgow\\[40pt]
A.E. Lindsay, \\
Department of Mathematics, University of Edinburgh\\[40pt]
J.R. Rosenberg \\
Division of Neuroscience and Biomedical Systems,
University of Glasgow
\end{center}
\end{slide}

%
%
\begin{slide}
\begin{center}
\textbf{Some general observations on \\ compartmental models}
\end{center}
\begin{enumerate}
\item
Motivation for compartmental modelling is the desire to reduce the
mathematical complexity inherent in a continuum description of a
neuron.

\item
A localised region of neuron (segment) and its input is described
by an elementary unit (often a simple circuit) called a compartment.

\item
The model is constructed by joining compartments in a
branching pattern corresponding to that of the neuron.

\item
Compartments can interact only with their nearest
neighbours.

\item
Compartments are connected together by enforcing conservation of current.
\end{enumerate}
\end{slide}

%
%
\begin{slide}
\centerline{\textbf{Traditional compartmental models - I}}

``\dots \, each lump of membrane becomes a compartment;
the rate  constants governing exchange between compartments
are proportional to the series conductance between them."
\begin{flushright}
Rall 1964
\end{flushright}
\centerline{\begin{mfpic}[1][1]{0}{250}{-30}{50}
\pen{1pt}
\lines{(30,10),(70,10),(70,-10),(30,-10)}
\dashed\lines{(30,10),(0,10)}
\dashed\lines{(30,-10),(0,-10)}
\rect{(80,10),(150,-10)}
\lines{(200,10),(160,10),(160,-10),(200,-10)}
\dashed\lines{(200,10),(230,10)}
\dashed\lines{(200,-10),(230,-10)}
\arrow\lines{(75,15),(75,5)}
\arrow\lines{(75,-15),(75,-5)}
\arrow\lines{(155,15),(155,5)}
\arrow\lines{(155,-15),(155,-5)}
\pen{2pt}
\headlen7pt
\lines{(40,15),(110,15)}
\lines{(40,-15),(110,-15)}
\lines{(120,15),(190,15)}
\lines{(120,-15),(190,-15)}
\tlabel[bc](75,20){\tiny \textsf{Membrane}}
\tlabel[bc](155,20){\tiny \textsf{Membrane}}
\end{mfpic}}

``Cable theory and compartmental models are
\underline{complementary} approaches $\cdots$ cable
theory regards dendrites as continuous cylindrical or conical
structures, whereas a compartmental model approximates a
continuous dendrite (or dendritic tree) as a set of resistively
coupled iso-potential regions.''
\begin{flushright}
Perkel and Mulloney 1978.
\end{flushright}
\vfill
\end{slide}

%
%
\begin{slide}
\centerline{\textbf{Traditional compartmental models - II}}

``Spatial discretization of this partial differential equation
(cable equation) is equivalent to reducing the spatially
distributed neuron to a set of connected compartments.''

``Spatially varying membrane current is represented by its value
at the center of the compartment. This is much less drastic than
the often heard statement that a compartment is assumed to be
iso-potential.''
\begin{flushright}
Hines and Carnevale 1997
\end{flushright}

``The compartmental approach \underline{replaces} the continuous
differential equations of the analytical model by a set of
ordinary differential equations. Thus, if the continuously
distributed system is divided into sufficiently small segments (or
compartments), one makes a negligibly small error by assuming that
each compartment is iso-potential and spatially uniform in its
properties.''

\begin{flushright}
Segev and Burke 1998
\end{flushright}
\vfill
\end{slide}

%
%
\begin{slide}
\begin{center}
\textbf{Summary of traditional models}
\end{center}

\begin{itemize}
\item
Two different definitions of a compartment. Despite these
conceptual differences, both approaches lead to the same
mathematical model.

\item
Each compartment has a single associated potential.

\item
Compartments cannot exist as independent entities, and therefore
seem to be an inappropriate choice for the fundamental building
blocks of a compartmental model. For example:-
\begin{enumerate}
\item
Half-compartments may be needed to model the behaviour of
terminals.

\item
The quantification of axial current requires two neighbouring
compartments.
\end{enumerate}
\end{itemize}
\end{slide}

%
%
\begin{slide}
\centerline{\textbf{How to define a compartment?}}

\textbf{Some guidance}

We form for ourselves images of symbols of external objects; and
the form that we give to them is such that the necessary
consequences of the images in thought are always images of the
necessary consequences in nature of the things pictured.
\begin{flushright}
Hertz: \emph{Principles of Mechanics}
\end{flushright}

The function of the model is to represent the necessity that
exists in nature by the logical necessity of the model. In the
case of a good model one parallels the other.
\begin{flushright}
Regnier: \emph{Les Infortunes de la Raison}
\end{flushright}
\end{slide}

%
%
\begin{slide}
\vfill

\begin{quotation}
\noindent In our view the primary weakness in traditional
compartmental models is that their compartments lack the
sensitivity necessary to reflect the location of input within a
segment. (We demonstrate this to be the case).
\end{quotation}

\begin{quotation}
\noindent It is also aesthetically unsatisfactory that traditional
compartments cannot exist as isolated entities by contrast with
the object they represent.
\end{quotation}
\vfil
\end{slide}

%
%
\begin{slide}
\centerline{\textbf{New compartmental model}}

\textbf{Idea}:- Assign two potentials to a compartment -- one
potential at each end of the length of dendrite (segment)
represented by the compartment.

\begin{itemize}
\item
Every dendritic section is a whole number of compartments.

\item
Neighbouring compartments share potentials at common boundaries.

\item
Compartments need not be iso-potential regions of dendrite.

\item
Model equations are constructed by enforcing conservation of
current at segment boundaries.
\end{itemize}
\vfill
\end{slide}

%
%
\begin{slide}
\centerline{\textbf{Partitioning of transmembrane current}}

\textbf{Problem}:- The new model requires all transmembrane
current acting on a segment to be divided between the axial
currents at the proximal and distal boundaries of the segment.

\centerline{\begin{mfpic}[1][1]{0}{400}{-60}{100}
\pen{1pt}
\headlen7pt
\rect{(0,0),(400,50)}
\arrow\lines{(5,25),(25,25)}
\tlabel[cl](30,25){\tiny $I_\mathrm{PD}+I_\mathrm{P}$}
\arrow\lines{(375,25),(395,25)}
\tlabel[cr](370,25){\tiny $I_\mathrm{PD}+I_\mathrm{D}$}
\arrow\lines{(240,50),(240,75)}
\tlabel[bc](230,85){\tiny $I_\mathrm{S}$}
\arrow\lines{(265,25),(245,25)}
\arrow\lines{(220,25),(235,25)}
\tlabel[tc](0,-10){$\lambda=0$}
\tlabel[tc](240,-10){$\lambda$}
\tlabel[tc](400,-10){$\lambda=1$}
\dashed\lines{(0,-30),(0,-60)}
\dashed\lines{(240,-30),(240,-60)}
\dashed\lines{(400,-30),(400,-60)}
\dotsize=1.5pt
\dotspace=4pt
\tlabel[cc](120,-45){$R_\mathrm{P}(\lambda)$}
\dotted\arrow\lines{(160,-45),(235,-45)}
\dotted\arrow\lines{(80,-45),(5,-45)}
\tlabel[cc](320,-45){$R_\mathrm{D}(\lambda)$}
\dotted\arrow\lines{(360,-45),(395,-45)}
\dotted\arrow\lines{(280,-45),(245,-45)}
\end{mfpic}}

The partitioning in this instance is:

\[
\begin{array}{ll}
\mbox{Towards }\lambda=0 \qquad& \frac{R_\mathrm{D}(\lambda)\,I_\mathrm{S}}
{R_\mathrm{P}(\lambda)+R_\mathrm{D}(\lambda)}\\[30pt]
\mbox{Towards }\lambda=1 \qquad& \frac{R_\mathrm{P}(\lambda)\,I_\mathrm{S}}
{R_\mathrm{P}(\lambda)+R_\mathrm{D}(\lambda)}
\end{array}
\]

\vfill
\end{slide}

%
%
\begin{slide}
\centerline{\textbf{Distributed transmembrane current}}

Consider a cylindrical dendritic segment of radius $a$, of
length $L$ filled with axoplasm of conductance $g_\mathrm{A}$
and with membrane of constant conductance $g_\mathrm{M}$.

A potential difference $V$ applied between the ends of
segment induces axial current flow
\[
\pi a^2 g_\mathrm{A} V/L
\]
and total transmembrane current flow
\[
2\pi a L g_\mathrm{M}\,(V/2)\,.
\]
The ratio of these currents is
\[
\frac{I_\mathrm{Transmembrane}}{I_\mathrm{Axial}}
=\frac{L^2 g_\mathrm{M}} {a
g_\mathrm{A}}=\frac{L^2}{a^2}\, \Bigg(
\frac{a g_\mathrm{M}}{g_\mathrm{A}}\Bigg).
\]
Typically $a g_\mathrm{M}/g_\mathrm{A}$ is small, say $\approx
10^{-5}$, which suggests that the transmembrane current acting on
a segment is small by comparison with axial current for ``short''
segments.
\end{slide}

%
%
\begin{slide}
\begin{center}
\textbf{Schematic illustration of a segment}
\end{center}

The Figure illustrates a dendritic segment of length $h$ (cm)
where $\lambda\in[0,1]$ indicates the fractional distance of a
point of the segment from its proximal end ($\lambda=0$).

\centerline{\begin{mfpic}[2][2]{-40}{140}{180}{310}
\headlen7pt
\pen{0.5pt}
\dotspace=4pt
\dotsize=1pt
\pen{1pt}
\dotspace=4pt
\dotsize=1.5pt
%
% LH cylinder
\parafcn[s]{-180,180,5}{(100-21*sind(t),240+28*cosd(t))}
\lines{(0,288),(100,268)}
\lines{(0,192),(100,212)}
%
% Partial cylinder on left
\dotted\parafcn[s]{0,180,5}{(36*sind(t),240+48*cosd(t))}
\parafcn[s]{0,180,5}{(-36*sind(t),240+48*cosd(t))}
%
% Annotation of LH cylinder
\dashed\arrow\lines{(0,240),(100,240)}
\tlabel[bl](50,250){\large $I_\mathrm{PD}$}
\tlabel[bc](0,250){$V_\mathrm{P}$}
\tlabel[cc](0,180){\large $\lambda=0$}
\tlabel[bc](0,295){\textsf{P}}
\arrow\lines{(0,235),(0,200)}
\tlabel[cr](-5,220){\textsf{$r_\mathrm{P}$}}
%
% Annotation of RH cylinder
\tlabel[bc](100,250){$V_\mathrm{D}$}
\tlabel[cc](100,180){\large $\lambda=1$}
\tlabel[cc](100,280){\textsf{D}}
\arrow\lines{(100,235),(100,216)}
\tlabel[cl](105,228){\textsf{$r_\mathrm{D}$}}
\arrow\lines{(60,228),(95,228)}
\arrow\lines{(40,228),(5,228)}
\tlabel[cc](50,228){$h$}
\end{mfpic}}

$I_\mathrm{PD}$ is axial current in the absence of transmembrane
current.
\end{slide}

%
%
\begin{slide}
\centerline{\textbf{Transmembrane current free solution}}
\begin{itemize}
\item
Segment membrane is modelled by the frustum of a cone of radius
\[
r(\lambda)=(1-\lambda)r_\mathrm{P}+\lambda r_\mathrm{D} \,,\qquad
\lambda\in[0,1]
\]
formed by rotating the straight line PD about the axis of the
dendrite

\item
Assuming axoplasm of constant axial conductance $g_\mathrm{A}$ and
that no transmembrane current acts on the segment then
\[
I_\mathrm{PD}= \frac{\pi g_\mathrm{A} r_\mathrm{P}
r_\mathrm{D}}{h}\,\big(\,V_\mathrm{P}-V_\mathrm{D}\,\big)
\]
and the membrane potential at point $\lambda$ is
\[
V(\lambda) = \frac{V_\mathrm{P}\,(1-\lambda)\,
r_\mathrm{P}+V_\mathrm{D}\,\lambda\,r_\mathrm{D}}
{(1-\lambda)\,r_\mathrm{P}+\lambda\,r_\mathrm{D}}\,.
\]
\end{itemize}
\end{slide}

%
%
\begin{slide}
\centerline{\textbf{Transmembrane current}}

Transmembrane current is usually described by the sum of four
distinct components:-
\begin{itemize}
\item
Capacitative current
\[
\int 2\pi r(x) \,c_\mathrm{M}\,\frac{\partial V}{\partial t}\,dx
\]

\item
Intrinsic voltage-dependent current (IVDC)
\[
\int 2\pi r(x)\,J_\mathrm{IVDC}(V)\,dx
\]

\item
Synaptic current
\[
\sum J_\mathrm{SYN}(V_\mathrm{syn})
\]

\item
Exogenous current
\[
\sum I_\mathrm{EX}(x,t)
\]
\end{itemize}
The integrals and summations in these expressions are calculated
over the segment.
\end{slide}

%
%
\begin{slide}
\centerline{\textbf{Conservation of current}}

Assume current $I_\mathrm{PD}+I_\mathrm{P}$ leaves the proximal
boundary of a segment towards its distal boundary and that current
$I_\mathrm{PD}+I_\mathrm{D}$ arrives at that distal boundary, then
\[
I_\mathrm{P}-I_\mathrm{D}=h\int_0^1 J(\lambda,t)\,d\lambda\,.
\]
where
\[
\begin{array}{rcl}
h J(\lambda,t) & = & \ds 2\pi h r(\lambda)\,c_\mathrm{M}(\lambda)\,
\frac{\partial V(\lambda,t)}{\partial t}\\[15pt]
&&\quad\ds+\;2\pi h r(\lambda)\,J_\mathrm{IVDC}(V(\lambda,t))\\[10pt]
&&\qquad \ds+\; \sum_k J_\mathrm{SYN}(V_\mathrm{syn})\,
\delta(\lambda-\lambda_k)\\[10pt]
&&\qquad\quad\ds+\;\sum_k I_\mathrm{EX}(t)\,
\delta(\lambda-\lambda_k)
\end{array}
\]
where $\lambda_k$ denotes the relative location of the $k^{th}$
synapse or exogenous input with respect to the proximal boundary
of the segment.
\end{slide}

%
%
\begin{slide}
\begin{center}
\textbf{Partitioning rule for \\ transmembrane current}
\end{center}

\textbf{Task}: We need expressions for $I_\mathrm{P}$ and
$I_\mathrm{D}$ that conserve current independently of the
constitutive forms for $J(\lambda,t)$.

\textbf{Idea}: Divide transmembrane current acting at point
$\lambda$ between the proximal and distal boundaries of
a segment in inverse proportion to the resistance of the segment
lying between the point $\lambda$ and that boundary.

This idea leads to the partitioning rule
\[
\begin{array}{rcl}
I_\mathrm{P} & = & \ds h\int_0^1 \frac{(1-\lambda)\,r_\mathrm{P}\,
J(\lambda,t)\,d\lambda}{(1-\lambda)\,r_\mathrm{P}
+\lambda\,r_\mathrm{D}}\,,\\[25pt]
-I_\mathrm{D} & = & \ds h\int_0^1
\frac{\lambda\,r_\mathrm{D}\,J(\lambda,t)\,d\lambda}
{(1-\lambda)\,r_\mathrm{P}+\lambda\,r_\mathrm{D}}\,.
\end{array}
\]

The rule is based on the observation that
\[
R_\mathrm{P}(\lambda) = \frac{\lambda h} {\pi g_\mathrm{A}
r_\mathrm{P} r(\lambda)}\,,\quad R_\mathrm{D}(\lambda) =
\frac{(1-\lambda)h} {\pi g_\mathrm{A} r_\mathrm{D}
r(\lambda)}\,,
\]
and that
\[
R_\mathrm{P}(\lambda)+R_\mathrm{D}(\lambda) =
\frac{h} {\pi g_\mathrm{A} r_\mathrm{P} r_\mathrm{D}}\,.
\]
\end{slide}

%
%
\begin{slide}
\begin{center}
\textbf{I - Exogenous point currents}
\end{center}
Let $\lambda_1,\cdots,\lambda_n$ be the locations of
exogenous current input $\mathcal{I}_1,\ \cdots\
\mathcal{I}_n$ to a segment.

The resulting transmembrane current is
\[
\sum_{k=1}^n \mathcal{I}_k(t)\,\delta(\lambda-\lambda_k)\,.
\]

The perturbations $I_\mathrm{P}$ and $I_\mathrm{D}$ to
$I_\mathrm{PD}$ are
\[
\begin{array}{rcl}
I_\mathrm{P} & = &\ds \sum_{k=1}^n\,
\frac{r_\mathrm{P}}{r_k}\,(1-\lambda_k)\,\mathcal{I}_k(t)\,,\\[25pt]
-I_\mathrm{D} & = &\ds \sum_{k=1}^n\,
\frac{r_\mathrm{D}}{r_k}\,\lambda_k\,\mathcal{I}_k(t)\,,
\end{array}
\]
where $r_k=(1-\lambda_k)\,r_\mathrm{P} +\lambda_k\,r_\mathrm{D}$.
\end{slide}

%
%
\begin{slide}
\begin{center}
\textbf{II - Synaptic point currents}
\end{center}
The current flow induced by $n$ synapses at points
$\lambda_1,\lambda_2, \cdots, \lambda_n$ on a segment of length
$h$ is represented by the Figure

\centerline{\qquad\begin{mfpic}[1.3][1]{0}{300}{-30}{60}
\pen{1pt}
\headlen7pt
%
% Sealed cable
\arrow\lines{(0,20),(25,20)}
\arrow\lines{(40,20),(65,20)}
\arrow\lines{(120,20),(145,20)}
\arrow\lines{(160,20),(185,20)}
\arrow\lines{(240,20),(265,20)}
\arrow\lines{(280,20),(305,20)}
%
%
\dotspace=4pt
\dotsize=2pt
\dotted\lines{(75,20),(110,20)}
\dotted\lines{(195,20),(230,20)}
%
% Nodes on sealed cable
\tlabel[cc](0,20){$\bullet$}
\tlabel[cc](40,20){$\bullet$}
\tlabel[cc](120,20){$\bullet$}
\tlabel[cc](160,20){$\bullet$}
\tlabel[cc](240,20){$\bullet$}
\tlabel[cc](280,20){$\bullet$}
\tlabel[cc](320,20){$\bullet$}
%
% Points on sealed cable
\tlabel[bc](40,30){\tiny $\lambda_1$}
\tlabel[bc](120,30){\tiny $\lambda_{k-1}$}
\tlabel[bc](160,30){\tiny $\lambda_k$}
\tlabel[bc](240,30){\tiny $\lambda_{n-1}$}
\tlabel[bc](280,30){\tiny $\lambda_n$}
%
\tlabel[cc](20,10){\tiny $I_1$}
\tlabel[cc](60,10){\tiny $I_2$}
\tlabel[cc](140,10){\tiny $I_k$}
\tlabel[cc](180,10){\tiny $I_{k+1}$}
\tlabel[cc](260,10){\tiny $I_n$}
\tlabel[cc](300,10){\tiny $I_{n+1}$}
%
% Sealed cable
\arrow\lines{(40,10),(40,-10)}
\tlabel[tc](40,-15){\tiny\textsf{$\mathcal{I}_1$}}
\arrow\lines{(120,10),(120,-10)}
\tlabel[tc](120,-15){\tiny\textsf{$\mathcal{I}_{k-1}$}}
\arrow\lines{(160,10),(160,-10)}
\tlabel[tc](160,-15){\tiny\textsf{$\mathcal{I}_k$}}
\arrow\lines{(240,10),(240,-10)}
\tlabel[tc](240,-15){\tiny\textsf{$\mathcal{I}_{n-1}$}}
\arrow\lines{(280,10),(280,-10)}
\tlabel[tc](280,-15){\tiny\textsf{$\mathcal{I}_n$}}
\end{mfpic}}

Synaptic input at $\lambda_k$ is modelled by the
constitutive law
\[
\mathcal{I}_k(t)=g_k(t)(V(\lambda_k,t)-E_k)
\]
where $E_k$ is the synaptic reversal potential and $g_k(t)$ is the
time course of the synaptic conductance.

The partitioning rule now gives
\[
\begin{array}{rcl}
I_\mathrm{P} & = &\ds \sum_{k=1}^n\,
\frac{r_\mathrm{P}}{r_k}\,(1-\lambda_k)\,g_k(t)\Bigg[
V(\lambda_k,t)-E_k\Bigg]\,,\\[25pt]
-I_\mathrm{D} & = &\ds \sum_{k=1}^n\,
\frac{r_\mathrm{D}}{r_k}\,\lambda_k\,g_k(t)\Bigg[V(\lambda_k,t)-E_k\Bigg]\,,
\end{array}
\]
but now $V(\lambda_k,t)$ is unknown.
\vfill
\end{slide}

%
%
\begin{slide}
\begin{center}
\textbf{Preliminary idea}
\end{center}
\begin{itemize}
\item
Estimate $V(\lambda_k,t)$ by the formula
\[
\widehat{V}_k(t)=V_\mathrm{P}(t)\,(1-\lambda_k)\,
\frac{r_\mathrm{P}}{r_k}+V_\mathrm{D}(t)\,\lambda_k\,
\frac{r_\mathrm{D}}{r_k}\,.
\]

\item For a single synapse at $\lambda_1$, the partitioning rule
with $V(\lambda_k,t)=\widehat{V}_k(t)$ gives
\[
\begin{array}{rcl}
I_\mathrm{P} & = &\ds \frac{r_\mathrm{P}}{r_1}\,(1-\lambda_1)\,g_1(t)
\Bigg[\widehat{V}_1(t)-E_1\Bigg]\,,\\[25pt]
-I_\mathrm{D} & = & \ds\frac{r_\mathrm{D}}{r_1}\,\lambda_1\,g_1(t)
\Bigg[\widehat{V}_1(t)-E_1\Bigg]\,.
\end{array}
\]

\item
In fact, the case of a single synaptic input at $\lambda_1$ has
exact solution
\[
\begin{array}{rcl}
I_\mathrm{P} & = & \ds\frac{r_\mathrm{P}}{r_1}\,
\frac{(1-\lambda_1)\,g_1(t)\Bigg[\widehat{V}_1(t)-E_1\Bigg]}
{1+\frac{\lambda_1(1-\lambda_1 )h g_1}{\pi g_\mathrm{A}
r^2_1}}\,,\\[35pt]
-I_\mathrm{D} & = &\ds\frac{r_\mathrm{D}}{r_1}\,
\frac{\lambda_1\,g_1(t)\Bigg[\widehat{V}_1(t)-E_1\Bigg]}
{1+\frac{\lambda_1(1-\lambda_1 )h g_1}{\pi g_\mathrm{A} r^2_1}}\,.
\end{array}
\]
The estimate $V(\lambda_k,t)=\widehat{V}_k(t)$ therefore
overstates the influence of the synapse.
\end{itemize}
\end{slide}

%
%
\begin{slide}
\begin{center}
\textbf{A more detailed analysis}
\end{center}
\textbf{Current conservation} at each synapse gives
\[
I_{k+1}+g_k(V_k-E_k) = I_k\,,\quad k=1,\cdots,n
\]
where $V_k$ is the potential at $\lambda_k$ and
\[
I_k = \frac{\pi g_\mathrm{A}r_{k-1}r_k}
{h(\lambda_k-\lambda_{k-1})}\,\big(V_{k-1}-V_k\,\big)
\]
for $k=1,\cdots,(n+1)$.

\vfil

\textbf{Solve the equation} relating $I_k$ to $V_k$ to get
\[
V_k = V_\mathrm{P} -\frac{h}{\pi g_\mathrm{A}}\,\sum_{j=1}^k \,
\frac{(\lambda_j-\lambda_{j-1})I_j}{r_{j-1}r_j}
\]
for $k=1,\cdots,(n+1)$. The case $k=n+1$ is the constraint
\[
\sum_{j=1}^{n+1} \,
\frac{(\lambda_j-\lambda_{j-1})I_j}{r_{j-1}r_j}=
\frac{h(V_\mathrm{P}-V_\mathrm{D})}{\pi g_\mathrm{A}}.
\]
\end{slide}

%
%
\begin{slide}
\textbf{Eliminate} $V_k$ from the current conservation condition
to obtain $n$ linear equations connecting $I_1$ to $I_{n+1}$. The
$(n+1)^{th}$ equation is the constraint.

\textbf{Re-express} the equations for $I_1$ to $I_{n+1}$ in terms
of $\widehat{I}_k=I_k-I_\mathrm{PD}$ where $\widehat{I}_k$ is the
perturbation in $I_k$ from $I_\mathrm{PD}$. The equations
satisfied by $\widehat{I}_0,\ \cdots \,\widehat{I}_{n+1}$ are

{\tiny
\[
\hskip-30pt\begin{array}{rcl} \ds\Big[1+\frac{\lambda_1 h g_1}{\pi
g_\mathrm{A} r_\mathrm{P}
r_1}\Big]\widehat{I}_1-\widehat{I}_2 & = & \mathcal{I}_1(t) \\[10pt]
\vdots & &\vdots\\
\ds\frac{g_k h}{\pi g_\mathrm{A}}\sum_{j=1}^{k-1}
\frac{(\lambda_j-\lambda_{j-1})\widehat{I}_j}{r_{j-1}r_j}+
\Big[1+\frac{(\lambda_k-\lambda_{k-1})hg_k}{\pi g_\mathrm{A}
r_{k-1}r_k}\Big]\widehat{I}_k-\widehat{I}_{k+1} & = & \mathcal{I}_k(t)\\[10pt]
\vdots & &\vdots\\
\ds\frac{g_n h}{\pi g_\mathrm{A}}\sum_{j=1}^{n-1}
\frac{(\lambda_j-\lambda_{j-1})\widehat{I}_j}{r_{j-1}r_j}+
\Big[1+\frac{(\lambda_n-\lambda_{n-1})hg_n}{\pi g_\mathrm{A}
r_{n-1}r_n}\Big]\widehat{I}_n-\widehat{I}_{n+1} & = & \mathcal{I}_n(t)\\[10pt]
\ds\sum_{j=1}^{n+1}\frac{(\lambda_j-\lambda_{j-1})
\widehat{I}_j}{r_{j-1}r_j} & = & 0
\end{array}
\]
} where $\mathcal{I}_k(t)=g_k(t)\,[\,\widehat{V}_k-E_k\,]$.
\end{slide}

%
%
\begin{slide}
These equations have matrix formulation
\[
A\,\widehat{I}+GD\,\widehat{I}=\mathcal{I}
\]
where $G$ and $D$ are $(n+1)\times(n+1)$ matrices and
\[
\begin{array}{rcl}
\widehat{I} & = & [\widehat{I}_1,\ \cdots\ ,
\widehat{I}_{n+1}]^\mathrm{T}\\[10pt]
\mathcal{I} & = & [\mathcal{I}_1,\ \cdots\ ,
\mathcal{I}_n,0\,]^\mathrm{T}
\end{array}
\]
The matrix $A$ is

{
\tiny
\[
\left[\begin{array}{ccccc}
1 & \hskip-9pt-1 &  0 & \cdots & 0 \\[5pt]
0 &  1 & \hskip-9pt-1 & \cdots & 0 \\[5pt]
0 &  0 &  1 & \cdots & 0 \\[5pt]
\cdots & \cdots & \cdots & \cdots & \cdots \\[5pt]
0 & 0 & 0 & \cdots & \hskip-9pt-1 \\[5pt]
\ds\frac{(\lambda_1-\lambda_0)}{r_0 r_1} &
\ds\frac{(\lambda_2-\lambda_1)}{r_1 r_2} &
\ds\frac{(\lambda_3-\lambda_2)}{r_2 r_3} &
\cdots & \ds\frac{(\lambda_{n+1}-\lambda_n)}{r_n r_{n+1}}
\end{array}\right]
\]
}
with inverse

{\tiny
\[
\left[\begin{array}{ccccccc}
(1-\lambda_1)\ds\frac{r_\mathrm{P}}{r_1} & (1-\lambda_2)\ds\frac{r_\mathrm{P}}{r_2} &
(1-\lambda_3)\ds\frac{r_\mathrm{P}}{r_3} &
\cdots & \cdots & r_\mathrm{P}r_\mathrm{D} \\[15pt]
 -\lambda_1\ds\frac{r_\mathrm{D}}{r_1} & (1-\lambda_2)\ds\frac{r_\mathrm{P}}{r_2} &
(1-\lambda_3)\ds\frac{r_\mathrm{P}}{r_3} &
\cdots & \cdots & r_\mathrm{P}r_\mathrm{D} \\[15pt]
 -\lambda_1\ds\frac{r_\mathrm{D}}{r_1} &  -\lambda_2\ds\frac{r_\mathrm{D}}{r_2} &
(1-\lambda_3)\ds\frac{r_\mathrm{P}}{r_3} &
\cdots & \cdots & r_\mathrm{P}r_\mathrm{D} \\[15pt]
\cdots & \cdots & \cdots & \cdots & \cdots & \cdots \\[15pt]
-\lambda_1\ds\frac{r_\mathrm{D}}{r_1} & -\lambda_2\ds\frac{r_\mathrm{D}}{r_2} &
-\lambda_3\ds\frac{r_\mathrm{D}}{r_3} & \cdots &
(1-\lambda_n)\ds\frac{r_\mathrm{P}}{r_n} & r_\mathrm{P}r_\mathrm{D} \\[15pt]
-\lambda_1\ds\frac{r_\mathrm{D}}{r_1} &
-\lambda_2\ds\frac{r_\mathrm{D}}{r_2} &
-\lambda_3\ds\frac{r_\mathrm{D}}{r_3} & \cdots &
-\lambda_n\ds\frac{r_\mathrm{D}}{r_n} & r_\mathrm{P}r_\mathrm{D}
\end{array}\right]\,.
\]
}
\end{slide}

%
%
\begin{slide}
\textbf{Exogenous point current}:

Here $G=0$ and therefore $\widehat{I}=A^{-1}\mathcal{I}$.
In fact, only the first ($I_\mathrm{P}$) and last ($I_\mathrm{D}$)
rows of $A^{-1}\mathcal{I}$ are needed.

\textbf{General synaptic point current}:

The first and last rows of $A^{-1}\mathcal{I}$ now give the
preliminary estimates of $I_\mathrm{P}$ and $I_\mathrm{D}$ (which
overestimate synaptic influence).

Solve by rewriting $A\,\widehat{I}+GD\,\widehat{I}=\mathcal{I}$ as
\[
A\,\widehat{I}^{\;(m+1)}=\mathcal{I}-GD\,\widehat{I}^{\;(m)}
\]
and iterate starting with $\widehat{I}^{\;(1)}=A^{-1}\mathcal{I}$.

Clearly the solution for $\widehat{I}$ has generic form
\[
\widehat{I} = \phi_1(t)V_\mathrm{P} + \phi_2(t)V_\mathrm{D}+\phi_3(t)
\]
where $\phi_1(t)$, $\phi_2(t)$ and $\phi_3(t)$ are functions of
time only.
\end{slide}

%
%
\begin{slide}
\begin{center}
\textbf{III - Capacitative current}
\end{center}
The contribution of capacitative current is estimated by
approximating the true membrane potential with
$\widehat{V}(\lambda,t)$ to obtain
\[
\begin{array}{rcl}
I^\mathrm{\,cap}_\mathrm{P} & = & 2\pi\, r_\mathrm{P} h
\ds\Bigg[r_\mathrm{P}\frac{dV_\mathrm{P}}{dt}
\int_0^1\frac{(1-\lambda)^2 c_\mathrm{M}(\lambda)\,d\lambda}
{(1-\lambda)\,r_\mathrm{P}+\lambda\,r_\mathrm{D}}\\[25pt]
&&\ds\quad+r_\mathrm{D}\frac{dV_\mathrm{D}}{dt}
\int_0^1 \frac{\lambda(1-\lambda)c_\mathrm{M}(\lambda)\,d\lambda}
{(1-\lambda)\,r_\mathrm{P}+\lambda\,r_\mathrm{D}}\Bigg],\\[25pt]
-I^\mathrm{\,cap}_\mathrm{D} & = & 2\pi\, r_\mathrm{D} h
\ds\Bigg[r_\mathrm{P}\frac{dV_\mathrm{P}}{dt}\int_0^1\,
\frac{\lambda(1-\lambda)c_\mathrm{M}(\lambda)\,d\lambda}
{(1-\lambda)\,r_\mathrm{P}+\lambda\,r_\mathrm{D}}\\[25pt]
&&\quad\ds+r_\mathrm{D}\frac{dV_\mathrm{D}}{dt}\int_0^1\,\frac{\lambda^2
c_\mathrm{M}(\lambda)\,d\lambda}
{(1-\lambda)\,r_\mathrm{P}+\lambda\,r_\mathrm{D}}\,\Bigg]\,.
\end{array}
\]
For a compartment with constant specific membrane capacitance in the
shape of a uniform right circular cylinder,
\[
\begin{array}{rcl}
I^\mathrm{\,cap}_\mathrm{P} & = & \ds\frac{C}{6}\,
\Bigg[\,2\frac{dV_\mathrm{P}}{dt}+\frac{dV_\mathrm{D}}{dt}\,\Bigg]\\[25pt]
-I^\mathrm{\,cap}_\mathrm{D} & = & \ds\frac{C}{6}\,
\Bigg[\,\frac{dV_\mathrm{P}}{dt}+2\frac{dV_\mathrm{D}}{dt} \,\Bigg]
\end{array}
\]
where $C$ is the total membrane capacitance of the segment.
\end{slide}

%
%
\begin{slide}
\begin{center}
\textbf{IV - Intrinsic voltage-dependent current}
\end{center}
IVDC is often described by the constitutive law
$J=g_\alpha(V)(V-E_\alpha)$ where $V$ is the membrane potential,
$E_\alpha$ is the reversal potential for species $\alpha$ and
$g_\alpha(V)$ is a voltage-dependent membrane conductance.

For a compartment with constant specific membrane capacitance in the
shape of a uniform right circular cylinder,
\[
\begin{array}{rcl}
I^\mathrm{\,ivdc}_\mathrm{P} & = & \ds\frac{\pi h r_\mathrm{P}}{6}
\,\Bigg[\,(3g_\mathrm{P}+g_\mathrm{D})(V_\mathrm{P}-E)\\[10pt]
&&\ds\qquad+\;(g_\mathrm{P}+g_\mathrm{D})(V_\mathrm{D}-E)\Bigg]\,,\\[20pt]
-I^\mathrm{\,ivdc}_\mathrm{D} & = & \ds\frac{\pi h r_\mathrm{D}}{6}
\,\Bigg[\,(g_\mathrm{P}+g_\mathrm{D})(V_\mathrm{P}-E)\\[10pt]
&&\qquad+\;(g_\mathrm{P}+3g_\mathrm{D})(V_\mathrm{D}-E)\Bigg].
\end{array}
\]
where $g_\mathrm{P}=g_\alpha(V_\mathrm{P})$ and $g_\mathrm{D}=g_\alpha(
V_\mathrm{D})$.
\end{slide}

%
%
\begin{slide}
\begin{center}
\textbf{Construction of the model equations}
\end{center}
The model (differential) equations are constructed as follows:-

\textbf{Equate} the expression for the axial current at the
distal boundary of one segment to that of the axial current
at the proximal boundary of the neighbouring segment.

\textbf{Apply} appropriate boundary conditions on the axial
current at the distal boundary of each terminal segment
(\emph{e.g.} current is zero).

\textbf{Assert} that the sum of the axial currents at the proximal
boundary of segments meeting at a branch point balances the axial
current at the distal boundary of the parent segment.

\textbf{Apply} a suitable somal boundary condition.
\end{slide}

%
%
\begin{slide}
The model differential equations are integrated using the
trapezoidal and midpoint quadratures, as appropriate.

The reorganised equations can usually be expressed in the form
\[
\begin{array}{ll}
\ds A_\mathrm{L}(t_{k+1},t_k)\,V^{(k+1)} = &
A_\mathrm{R}(t_{k+1},t_k)\,V^{(k)}\\[10pt]
&\quad+B(t_{k+1},t_k)+O(h^3).
\end{array}
\]
where $A_\mathrm{L}$ and $A_\mathrm{R}$ are sparse matrix with
structural form that of the connectivity matrix of the neuron.
\end{slide}

%
%
\begin{slide}
\begin{center}
\textbf{Model Neuron}
\end{center}
The accuracy of the new and traditional compartmental models is
compared using a branched neuron for which the continuum model has
a closed form expression for the membrane potential in response to
input.
\[
\begin{array}{c}
$\begin{mfpic}[1.3][1.3]{0}{220}{-20}{220}
\pen{2pt}
\dotsize=1pt
\dotspace=3pt
\lines{(-5,100),(5,110),(15,100),(5,90),(-5 ,100)}
% Upper dendrite
% Root branch
\dotted\lines{(5,115),(15,170),(20,170)}
\lines{(20.0,160),(36.7,160)}
\tlabel[tc](28.4,150){\textsf{(a)}}
% Level 1
\lines{(50.0,190),(88.3,190)}
\tlabel[bc](75,200){\textsf{(c)}}
\lines{(50.0,130),(91.0,130)}
\tlabel[tc](75,120){\textsf{(d)}}
\dotted\lines{(36.7,160),(45,200),(55,200)}
\dotted\lines{(36.7,160),(45,120),(55,120)}
% Level 2
\lines{(100.0,210),(153.2,210)}
\lines{(100.0,190),(153.2,190)}
\lines{(100.0,170),(153.2,170)}
\tlabel[cl](160,210){\textsf{(g)}}
\tlabel[cl](160,190){\textsf{(g)}}
\tlabel[cl](160,170){\textsf{(g)}}
\dotted\lines{(88.3,190),(95,220),(105,220)}
\dotted\lines{(88.3,190),(95,160),(105,160)}
\lines{(100.0,140),(165.1,140)}
\lines{(100.0,120),(165.1,120)}
\dotted\lines{(91.0,130),(95,150),(105,150)}
\dotted\lines{(91.0,130),(95,110),(105,110)}
\tlabel[cl](175,140){\textsf{(h)}}
\tlabel[cl](175,120){\textsf{(h)}}
%
% Lower dendrite
% Root branch
\lines{(20.0,40),(58.0,40)}
\dotted\lines{(5,85),(15,30),(25,30)}
\tlabel[bc](39,50){\textsf{(b)}}
% Level 1
\lines{(70.0,70),(133.1,70)}
\lines{(70.0,10),(127.1,10)}
\dotted\lines{(58,40),(66.5,80),(76.5,80)}
\dotted\lines{(58,40),(66.5,0),(76.5,0)}
\tlabel[bc](105,80){\textsf{(e)}}
\tlabel[tc](105,0){\textsf{(f)}}
% Level 2
\lines{(145,80),(195.1,80)}
\lines{(145,60),(195.1,60)}
\dotted\lines{(133.1,70),(140,90),(150,90)}
\dotted\lines{(133.1,70),(140,50),(150,50)}
\tlabel[cl](205,80){\textsf{(i)}}
\tlabel[cl](205,60){\textsf{(i)}}
\lines{(140,30),(179.6,30)}
\lines{(140,10),(179.6,10)}
\lines{(140,-10),(179.6,-10)}
\dotted\lines{(127.1,10),(134,40),(144,40)}
\dotted\lines{(127.1,10),(134,-20),(144,-20)}
\tlabel[cl](190,30){\textsf{(j)}}
\tlabel[cl](190,10){\textsf{(j)}}
\tlabel[cl](190,-10){\textsf{(j)}}
\end{mfpic}$
\end{array}
\]
The diameters and lengths of the dendritic sections are organised
so that $l/\sqrt{r}$ is fixed for all children of each branch point.
\end{slide}

%
%
\begin{slide}
It can be proved that the solution of any well-configured
compartmental model of a neuron converges to the solution of the
continuum model of that neuron in the limit as the maximum segment
size tends to zero.

The accuracy of the new compartmental model will be compared with
that of a traditional compartmental model by measuring the
closeness with which the solutions of both models estimate the
exact solution for the somal potential of the continuum model.
\end{slide}

%
%
\begin{slide}
\centerline{\textbf{Simulation exercises}}

\begin{itemize}
\item 75 exogenous point inputs each of strength
$2\times10^{-5}\,\mu$A were distributed randomly over the branched
model neuron.

\item The difference between the exact somal potential and its
value computed by each compartmental model was determined at one
millisecond intervals for the first 10 milliseconds of the
simulation.

\item  Each difference was divided by the (known) exact potential
of the soma at that time to get relative errors for that model.

\item For a fixed number of compartments, each simulation exercise
consisted of 2000 repetitions of the procedure for each model (for
identical input).
\end{itemize}
\end{slide}

%
%
\begin{slide}
\begin{center}
\textbf{Results}
\end{center}

\vfil

\centerline{\begin{mfpic}[84][36]{0.4}{3}{-7.2}{1.3}
\headlen7pt
\pen{1pt}
\dotspace=4pt
\dotsize=1.5pt
%
% x-axis
\tlabel[br](3.0,0.7){\tiny\textsf{$\log_{10}(\mbox{No. Compartments})$}}
\lines{(1.0,0),(3.0,0)}
\lines{(1.5,0),(1.5,-0.2)}
\lines{(2.0,0),(2.0,-0.2)}
\lines{(2.5,0),(2.5,-0.2)}
\lines{(3.0,0),(3.0,-0.2)}
\tlabel[bc](1.0,0.3){\tiny\textsf{1.0}}
\tlabel[bc](1.5,0.3){\tiny\textsf{1.5}}
\tlabel[bc](2.0,0.3){\tiny\textsf{2.0}}
\tlabel[bc](2.5,0.3){\tiny\textsf{2.5}}
\tlabel[bc](3.0,0.3){\tiny\textsf{3.0}}
% y-axis
\tlabel[bc](0.5,-6){\rotatebox{90}{\tiny\textsf{$\log_{10}(\mbox{Mean
relative error})$}}} \lines{(1,0),(1,-7)}
\lines{(1.0,-1.0),(1.05,-1.0)}
\lines{(1.0,-2.0),(1.05,-2.0)}
\lines{(1.0,-3.0),(1.05,-3.0)}
\lines{(1.0,-4.0),(1.05,-4.0)}
\lines{(1.0,-5.0),(1.05,-5.0)}
\lines{(1.0,-6.0),(1.05,-6.0)}
\lines{(1.0,-7.0),(1.05,-7.0)}
\tlabel[cr](0.95,-0.0){\tiny\textsf{0.0}}
\tlabel[cr](0.95,-1.0){\tiny\textsf{-1.0}}
\tlabel[cr](0.95,-2.0){\tiny\textsf{-2.0}}
\tlabel[cr](0.95,-3.0){\tiny\textsf{-3.0}}
\tlabel[cr](0.95,-4.0){\tiny\textsf{-4.0}}
\tlabel[cr](0.95,-5.0){\tiny\textsf{-5.0}}
\tlabel[cr](0.95,-6.0){\tiny\textsf{-6.0}}
\tlabel[cr](0.95,-7.0){\tiny\textsf{-7.0}}
%
\lines{(2,-1),(2.3,-1)}
\tlabel[cl](2.4,-1){\tiny\textsf{New Model}}
\dashed\lines{(2,-1.8),(2.3,-1.8)}
\tlabel[cl](2.4,-1.8){\tiny\textsf{NEURON}}
%
% Mean values at t=10
\dashed\lines{(1.2,-2.494),(3.0,-4.60)}
\lines{(1.2,-2.686),(3.0,-6.466)}
\end{mfpic}}

\vfil

\centerline{\begin{mfpic}[84][36]{0.4}{3}{-7.2}{1.3}
\headlen7pt
\pen{1pt}
\dotspace=4pt
\dotsize=1.5pt
%
% x-axis
\tlabel[br](3.0,0.7){\tiny\textsf{$\log_{10}(\mbox{No Compartments})$}}
\lines{(1.0,0),(3.0,0)}
\lines{(1.5,0),(1.5,-0.2)}
\lines{(2.0,0),(2.0,-0.2)}
\lines{(2.5,0),(2.5,-0.2)}
\lines{(3.0,0),(3.0,-0.2)}
\tlabel[bc](1.0,0.3){\tiny\textsf{1.0}}
\tlabel[bc](1.5,0.3){\tiny\textsf{1.5}}
\tlabel[bc](2.0,0.3){\tiny\textsf{2.0}}
\tlabel[bc](2.5,0.3){\tiny\textsf{2.5}}
\tlabel[bc](3.0,0.3){\tiny\textsf{3.0}}
% y-axis
\tlabel[bc](0.5,-6){\rotatebox{90}{\tiny\textsf\tiny{$\log_{10}(\mbox{Standard
Dev.})$}}} \lines{(1,0),(1,-7)}
\lines{(1.0,-1.0),(1.05,-1.0)}
\lines{(1.0,-2.0),(1.05,-2.0)}
\lines{(1.0,-3.0),(1.05,-3.0)}
\lines{(1.0,-4.0),(1.05,-4.0)}
\lines{(1.0,-5.0),(1.05,-5.0)}
\lines{(1.0,-6.0),(1.05,-6.0)}
\lines{(1.0,-7.0),(1.05,-7.0)}
\tlabel[cr](0.95,-0.0){\tiny\textsf{0.0}}
\tlabel[cr](0.95,-1.0){\tiny\textsf{-1.0}}
\tlabel[cr](0.95,-2.0){\tiny\textsf{-2.0}}
\tlabel[cr](0.95,-3.0){\tiny\textsf{-3.0}}
\tlabel[cr](0.95,-4.0){\tiny\textsf{-4.0}}
\tlabel[cr](0.95,-5.0){\tiny\textsf{-5.0}}
\tlabel[cr](0.95,-6.0){\tiny\textsf{-6.0}}
\tlabel[cr](0.95,-7.0){\tiny\textsf{-7.0}}
%
\lines{(2,-1),(2.3,-1)}
\tlabel[cl](2.4,-1){\tiny\textsf{New Model}}
\dashed\lines{(2,-1.8),(2.3,-1.8)}
\tlabel[cl](2.4,-1.8){\tiny\textsf{NEURON}}
%
% Standard deviations at t=10
\dashed\lines{(1.2,-2.664),(3.0,-4.680)}
\lines{(1.2,-3.169),(3.0,-7.021)}
\end{mfpic}}
\vfill
\end{slide}

%
%
\begin{slide}
\centerline{\textbf{Concluding remarks}}
\begin{itemize}
\item At each level of discretisation the new compartmental model
always performs better that the traditional model.

\item This improvement in accuracy is achieved without a
comparable increase in computational effort. In the simulation
exercises described here, there was no discernible difference in
computational effort.

\item This difference in accuracy can be attributed in large part
to the fact that the new model is more sensitive to the location
of point input than the traditional model.

\end{itemize}

\end{slide}

%
%
\begin{slide}
\centerline{\textbf{Regression lines}}

\textbf{Mean regressions}

NEURON ($R^2=97.4\%$ (adj))
\[
\log_{10}(\mbox{MRE})=-1.09-1.17\log_{10}(\mbox{Nodes}).
\]

New Model ($R^2=99.5\%$ (adj))
\[
\log_{10}(\mbox{MRE})=-0.17-2.10\log_{10}(\mbox{Nodes}).
\]

\textbf{Standard deviations}

NEURON ($R^2=98.7\%$ (adj))
\[
\log_{10}(\mbox{SD})=-1.32-1.12\log_{10}(\mbox{Nodes}).
\]

New Model ($R^2=99.4\%$ (adj))
\[
\log_{10}(\mbox{SD})=-0.60-2.14\log_{10}(\mbox{Nodes})
\]
\end{slide}
\closegraphsfile
\end{document}

%
%
\begin{slide}
\begin{center}
\textbf{Some additional comments}
\end{center}
Compartmental models of a dendrite begin with a subdivision of its
sections into contiguous segments which, in turn, define the
compartments of the mathematical model.

\vfil

Although the traditional and new compartmental models assign
membrane potentials throughout a dendrite in different ways, both
approaches use identical numbers of unknown potentials.

\vfil

The traditional and new compartmental models both involve nearest
neighbour interactions, and give rise to families of differential
equations that are structurally identical.

\vfil

The numerical solution of the ODE's requires the solution of a
sparse matrix system in which the controlling matrix is time
dependent. The equations can be efficiently solved by an iterative
procedures.
\vfill
\end{slide}

%
%
\begin{slide}
\begin{center}
\textbf{Analytical solution}
\end{center}
The deviation $V(t)$ of the somal potential from its resting value
as a result of distributed current $I(x,t)$ on a uniform
cylindrical dendrite of radius $a$ and length $l$ attached to a
soma is
\[
V(t)=e^{-t/\tau}\,\Big[\,\phi_0(t)+\sum_\beta\;\phi_\beta(t)
e^{-\beta^2 t/L^2\tau}\,\cos\beta\,\Big]
\]
where, in the usual notation,
\[
L=l\,\sqrt{\ds\frac{2 g_\mathrm{M}}{a g_\mathrm{A}}}
\]
and $\tau$ is the time constant of the somal and dendritic
membranes (assumed identical). The summation is taken over all the
non-negative solutions $\beta$ of the transcendental equation
$\tan\beta+\gamma\beta=0$ where
\[
\gamma=\frac{\mbox{Somal membrane leakage conductance}}
{\mbox{Dendritic membrane leakage conductance}}
\]

\pagebreak[4]

In the special case of a neuron, initially at its resting
potential, and stimulated by point currents $I_k(t)$ at distance
$x_k$ from the soma of the uniform cylinder ($k=1,\cdots,n$), the
coefficient functions $\phi_0$ and $\phi_\beta$ satisfy
\[
\begin{array}{rcl}
\ds\frac{d\phi_0}{dt} & = &\ds -\frac{e^{t/\tau}}
{C_\mathrm{D}+C_\mathrm{S}}\,\Bigg[\,
I_\mathrm{S}(t)+\sum_{k=1}^n\;I_k(t)\,\Bigg]\,,\\[25pt]
\ds\frac{d\phi_\beta}{dt} & = &
\ds-\frac{2e^{(1+\beta^2/L^2)t/\tau}}
{C_\mathrm{D}+C_\mathrm{S}\cos^2\beta}\,\times\\[20pt]
&&\hskip-20pt\ds\Bigg[\, \sum_{k=1}^n
\;I_k(t)\cos\beta\big(1-x_k/l\big)
+\cos\beta\,I_\mathrm{S}(t)\,\Bigg]\,.
\end{array}
\]
The parameters $C_\mathrm{D}$ and $C_\mathrm{S}$ denote
respectively the total membrane capacitances of the soma and
dendrite, and $I_\mathrm{S}(t)$ is the current supplied to the
soma.
\end{slide}

%
%
\begin{slide}
\begin{center}
\textbf{Use of the analytical solution}
\end{center}

New and traditional compartmental models (the latter represented
by the NEURON simulator) were compared by estimating the accuracy
with which both models computed the time course of the potential
at the soma of the model neuron under the action of large scale
exogenous input on its dendrites. Facts.

\textbf{The connected cable} of the model neuron is equivalent to
a cylinder (Rall cylinder).

\textbf{The effect} of any configuration of exogenous input at the
soma of the model neuron (assumed to be a sphere of diameter
$40\,\mu$m) is exactly representable at the soma of the equivalent
cylinder by a suitable configuration of exogenous input on that
cylinder.

\textbf{Each input} on the model neuron is mapped to an input on
the equivalent cylinder of identical strength and lying at the
same electrotonic distance from the soma of the equivalent
cylinder as the original input lies from the soma of the model
neuron. \vfill
\end{slide}
